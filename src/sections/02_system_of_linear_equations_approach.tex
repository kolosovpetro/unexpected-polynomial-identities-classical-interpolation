In 2018, Albert Tkaczyk published two papers~\cite{tkaczyk2018problem, tkaczyk2018continuation}
presenting analogous identities for polynomials $n^5, \; n^7$ and $n^9$
derived in a manner similar to $n^3 = \sum_{k=1}^{n} 6k(n-k) + 1$.
Further, his results were polished and published in \textit{Mathematical gazette}
~\cite{unusual_identity_for_odd_powers}.
That's kinda funny, polished results of Polish person, ha-ha.
Tkaczyk assumed that the identity for $n^5$ takes the following explicit form
\begin{align*}
    n^5 = \sum_{k=1}^{n} \left[ A k^2(n-k)^2 + Bk(n-k) + C \right]
\end{align*}
where $A,B,C$ are yet-unknown coefficients.
We denote $A,B,C$ as $\coeffA{2}{0}, \coeffA{2}{1}, \coeffA{2}{2}$
to reach the compact form of double sum
\begin{align*}
    n^5 = \sum_{k=1}^{n} \sum_{r=0}^{2} \coeffA{2}{r} k^r (n-k)^r
\end{align*}
By observing the equation above, the potential form of generalized odd-power identity becomes more obvious.
One important note to add here, we define $0^x = 1$ for all $x$, see~\cite[~p. 162]{graham1994concrete}.
This is because when $k=n$ and $r=0$ the term $k^r (n-k)^r = n^0 \cdot 0^0$, thus we define $0^x = 1$
for all $x$.

To evaluate the set of coefficients $\coeffA{2}{0}, \coeffA{2}{1}, \coeffA{2}{2}$
we construct and solve a certain system of linear equations, which is
built as follows
\begin{align*}
    n^5 = \coeffA{2}{0} \sum_{k=1}^{n} k^0 (n-k)^0 + \coeffA{2}{1} \sum_{k=1}^{n} k^1 (n-k)^1 + \coeffA{2}{2} \sum_{k=1}^{n} k^2 (n-k)^2
\end{align*}
By expanding the sums $\sum_{k=1}^{n} k^r (n-k)^r$ using Faulhaber's formula~\cite{beardon1996sums}, we get
an equation
\begin{equation*}
    \coeffA{2}{0} n
    + \coeffA{2}{1} \left[ \frac{1}{6} (n^3-n) \right]
    + \coeffA{2}{2} \left[ \frac{1}{30} (n^5-n) \right] - n^5 = 0
\end{equation*}
By multiplying by $30$ both right-hand side and left-hand side, we get
\begin{equation*}
    30 \coeffA{2}{0} n + 5 \coeffA{2}{1} (n^3-n) + \coeffA{2}{2} (n^5-n) - 30n^5 = 0
\end{equation*}
By expanding the brackets and rearranging the terms
\begin{equation*}
    30 \coeffA{2}{0} - 5 \coeffA{2}{1} n + 5 \coeffA{2}{1} n^3 - \coeffA{2}{2} n + \coeffA{2}{2} n^5 - 30n^5 = 0
\end{equation*}
By combining the common terms, we obtain
\begin{equation*}
    n (30 \coeffA{2}{0} - 5 \coeffA{2}{1} - \coeffA{2}{2}) + 5 \coeffA{2}{1} n^3 + n^5 (\coeffA{2}{2} - 30) = 0
\end{equation*}
Therefore,
\begin{equation*}
    \begin{cases}
        30 \coeffA{2}{0} - 5 \coeffA{2}{1} - \coeffA{2}{2} &= 0 \\
        \coeffA{2}{1} &= 0 \\
        \coeffA{2}{2} - 30 &= 0
    \end{cases}
\end{equation*}
By solving the system above, we evaluate the coefficients $\coeffA{2}{0}, \coeffA{2}{1}, \coeffA{2}{2}$
\begin{equation*}
    \begin{cases}
        \coeffA{2}{2} &= 30 \\
        \coeffA{2}{1} &= 0 \\
        \coeffA{2}{0} &= 1
    \end{cases}
\end{equation*}
Thus, the identity for $n^5$
\begin{equation*}
    n^5 = \sum_{k=1}^{n} 30k^2(n-k)^2 + 1
\end{equation*}
Again, the terms $30k^2(n-k)^2 + 1$ are symmetric over $k$.
Let be $T_2 (n,k) = 30k^2(n-k)^2 + 1$ then
\begin{align*}
    T_2 (n,k) = T_2 (n,n-k)
\end{align*}
By arranging the values of $T_{2} (n,k)$ as a triangular array, we see that the identity for $n^5$ is indeed true
\begin{table}[H]
    \setlength\extrarowheight{-6pt}
    \begin{tabular}{c|cccccccc}
        $n/k$ & 0 & 1    & 2    & 3    & 4    & 5    & 6    & 7 \\
        \hline
        0     & 1 &      &      &      &      &      &      &   \\
        1     & 1 & 1    &      &      &      &      &      &   \\
        2     & 1 & 31   & 1    &      &      &      &      &   \\
        3     & 1 & 121  & 121  & 1    &      &      &      &   \\
        4     & 1 & 271  & 481  & 271  & 1    &      &      &   \\
        5     & 1 & 481  & 1081 & 1081 & 481  & 1    &      &   \\
        6     & 1 & 751  & 1921 & 2431 & 1921 & 751  & 1    &   \\
        7     & 1 & 1081 & 3001 & 4321 & 4321 & 3001 & 1081 & 1
    \end{tabular}
    \caption{Values of $30k^2(n-k)^2 + 1$.
    See the OEIS entry \href{https://oeis.org/A300656}{\texttt{A300656}}
    ~\cite{oeis_numerical_triangle_row_sums_give_fifth_powers}.}
    \label{tab:row-sums-gives-fifth-power}
\end{table}

The following recurrence holds for $T_2 (n,k)$
\begin{align*}
    T_2 (n, k) = 3T_2(n-1, k) - 3T_2(n-2, k) + T_2(n-3, k)
\end{align*}
Which is indeed true because
\begin{align*}
    T_2 (6,2) = 3 \cdot 1081 - 3 \cdot 481 + 271 = 1921
\end{align*}

Thus, our curious learner who stated the question~\eqref{question:higher_powers} got the answer, yet again
green light was lit.
This time, the answer contained even more than methodology to find
a set of coefficients $\coeffA{2}{0}, \coeffA{2}{1}, \ldots, \coeffA{2}{2}$ --- it contained a generic
form of odd power identity $n^{2m+1}$ for any natural $m$.
Hence, the part of questions-answers we discuss so far ends here, it is time to state a conjecture.
\begin{conjecture}
    \label{conj:odd-power-identity}
    There is a set of coefficients $\coeffA{m}{0}, \coeffA{m}{1}, \ldots, \coeffA{m}{m}$ such that
    \begin{align*}
        n^{2m+1} = \sum_{r=0}^{m} \sum_{k=1}^{n} \coeffA{m}{r} k^r (n-k)^r
    \end{align*}
\end{conjecture}
We already know that to identify the coefficients we have to build and solve
a certain system of linear equations, however, we cannot perform any kind of induction
on that, hence conjecture cannot be proven just by building and solving endless
systems of linear equations.
There must be a formula that evaluates the set of coefficients $\coeffA{m}{0}, \coeffA{m}{1}, \ldots, \coeffA{m}{m}$
for every non-negative integer $m$ --- our young investigator thought.
