In 2018, Albert Tkaczyk shared two preprints~\cite{tkaczyk2018problem, tkaczyk2018continuation}
addressing the problem~\eqref{question:higher_powers}.
These preprints discussed the cases for  $n^5, \; n^7$ and $n^9$ in context of problem~\eqref{question:higher_powers}.
Furthermore, these results were polished and published in \textit{Mathematical gazette}
~\cite{unusual_identity_for_odd_powers} in 2021.
Tkaczyk assumed that the identity for $n^5$ takes the following explicit form
\begin{align*}
    n^5 = \sum_{k=1}^{n} \left[ A k^2(n-k)^2 + Bk(n-k) + C \right]
\end{align*}
where $A,B,C$ are unknown coefficients.
We denote $A,B,C$ as $\coeffA{2}{0}, \coeffA{2}{1}, \coeffA{2}{2}$
to get the compact form of double sum
\begin{align*}
    n^5 = \sum_{k=1}^{n} \sum_{r=0}^{2} \coeffA{2}{r} k^r (n-k)^r
\end{align*}
By observing the equation above, the form of generalized odd-power identity becomes more clear and obvious.
One important note to add here, we define $x^0 = 1$ for all $x$, see~\cite[~p. 162]{graham1994concrete}.
This is because when $k=n$ and $r=0$ the term $k^r (n-k)^r = n^0 \cdot 0^0$, thus we define $x^0 = 1$
for all $x$.

To evaluate the set of coefficients $\coeffA{2}{0}, \coeffA{2}{1}, \coeffA{2}{2}$
we have to build and solve a system of linear equations
\begin{align*}
    n^5 = \coeffA{2}{0} \sum_{k=1}^{n} k^0 (n-k)^0 + \coeffA{2}{1} \sum_{k=1}^{n} k^1 (n-k)^1 + \coeffA{2}{2} \sum_{k=1}^{n} k^2 (n-k)^2
\end{align*}
By expanding the sums $\sum_{k=1}^{n} k^r (n-k)^r$ using Faulhaber's formula~\cite{beardon1996sums}, we get
\begin{equation*}
    \coeffA{2}{0} n
    + \coeffA{2}{1} \left[ \frac{1}{6} (n^3-n) \right]
    + \coeffA{2}{2} \left[ \frac{1}{30} (n^5-n) \right] - n^5 = 0
\end{equation*}
By multiplying by $30$ both right-hand side and left-hand side, we get
\begin{equation*}
    30 \coeffA{2}{0} n + 5 \coeffA{2}{1} (n^3-n) + \coeffA{2}{2} (n^5-n) - 30n^5 = 0
\end{equation*}
By expanding the brackets and rearranging the terms
\begin{equation*}
    30 \coeffA{2}{0} - 5 \coeffA{2}{1} n + 5 \coeffA{2}{1} n^3 - \coeffA{2}{2} n + \coeffA{2}{2} n^5 - 30n^5 = 0
\end{equation*}
By combining the common terms, we obtain
\begin{equation*}
    n (30 \coeffA{2}{0} - 5 \coeffA{2}{1} - \coeffA{2}{2}) + 5 \coeffA{2}{1} n^3 + n^5 (\coeffA{2}{2} - 30) = 0
\end{equation*}
Therefore,
\begin{equation*}
    \begin{cases}
        30 \coeffA{2}{0} - 5 \coeffA{2}{1} - \coeffA{2}{2} &= 0 \\
        \coeffA{2}{1} &= 0 \\
        \coeffA{2}{2} - 30 &= 0
    \end{cases}
\end{equation*}
By solving the system above, we evaluate the coefficients $\coeffA{2}{0}, \coeffA{2}{1}, \coeffA{2}{2}$
\begin{equation*}
    \begin{cases}
        \coeffA{2}{2} &= 30 \\
        \coeffA{2}{1} &= 0 \\
        \coeffA{2}{0} &= 1
    \end{cases}
\end{equation*}
Thus, the identity for $n^5$
\begin{equation*}
    n^5 = \sum_{k=1}^{n} 30k^2(n-k)^2 + 1
\end{equation*}
Again, the terms $30k^2(n-k)^2 + 1$ are symmetric over $k$.
Let be $T_2 (n,k) = 30k^2(n-k)^2 + 1$ then
\begin{align*}
    T_2 (n,k) = T_2 (n,n-k)
\end{align*}
By arranging the values of $T_{2} (n,k)$ as tabular, we see that the identity for $n^5$ is indeed true
\begin{table}[H]
    \setlength\extrarowheight{-6pt}
    \begin{tabular}{c|cccccccc}
        $n/k$ & 0 & 1    & 2    & 3    & 4    & 5    & 6    & 7 \\
        \hline
        0     & 1 &      &      &      &      &      &      &   \\
        1     & 1 & 1    &      &      &      &      &      &   \\
        2     & 1 & 31   & 1    &      &      &      &      &   \\
        3     & 1 & 121  & 121  & 1    &      &      &      &   \\
        4     & 1 & 271  & 481  & 271  & 1    &      &      &   \\
        5     & 1 & 481  & 1081 & 1081 & 481  & 1    &      &   \\
        6     & 1 & 751  & 1921 & 2431 & 1921 & 751  & 1    &   \\
        7     & 1 & 1081 & 3001 & 4321 & 4321 & 3001 & 1081 & 1
    \end{tabular}
    \caption{Values of $30k^2(n-k)^2 + 1$.
    See the OEIS entry \href{https://oeis.org/A300656}{\texttt{A300656}}
    ~\cite{oeis_numerical_triangle_row_sums_give_fifth_powers}.}
    \label{tab:row-sums-gives-fifth-power}
\end{table}

The following recurrence holds for $T_2 (n,k)$
\begin{align*}
    T_2 (n, k) = 3T_2(n-1, k) - 3T_2(n-2, k) + T_2(n-3, k)
\end{align*}
Which is indeed true because
\begin{align*}
    T_2 (6,2) = 3 \cdot 1081 - 3 \cdot 481 + 271 = 1921
\end{align*}

Thus, our curious learner obtained a solution for the problem~\eqref{question:higher_powers}.
This time, the solution contained even more than methodology to evaluate
the coefficients $\coeffA{2}{0}, \coeffA{2}{1}, \ldots, \coeffA{2}{2}$ ---
it pointed out to a general form of identity for odd powers.

Hence, let be a conjecture
\begin{conjecture}
    \label{conj:odd-power-identity}
    For every integer $m \geq 0$,
    there is a set of coefficients $\coeffA{m}{0}, \coeffA{m}{1}, \ldots, \coeffA{m}{m}$ such that
    \begin{align*}
        n^{2m+1} = \sum_{r=0}^{m} \sum_{k=1}^{n} \coeffA{m}{r} k^r (n-k)^r
    \end{align*}
\end{conjecture}
We know already that to evaluate the coefficients $\coeffA{m}{r}$ we have to build and solve
a system of linear equations.
However, we cannot apply any sort of induction over these systems of linear equations.
The conjecture~\eqref{conj:odd-power-identity} cannot be proven by building and solving countless
systems of linear equations for any integer $m\geq 0$.
There must be a formula that evaluates the set of coefficients $\coeffA{m}{0}, \coeffA{m}{1}, \ldots, \coeffA{m}{m}$
for every non-negative integer $m$ --- our young investigator thought.
