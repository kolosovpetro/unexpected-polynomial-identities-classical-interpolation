Thus, the conjecture~\eqref{conj:odd-power-identity} is true

\begin{theorem}[Odd power identity]
    \label{theorem:odd-power-identity}
    There is a set of coefficients $\coeffA{m}{0}, \coeffA{m}{1}, \ldots, \coeffA{m}{m}$ such that
    \begin{align*}
        n^{2m+1} = \sum_{r=0}^{m} \sum_{k=1}^{n} \coeffA{m}{r} k^r (n-k)^r
    \end{align*}
\end{theorem}
In explicit form
\begin{align*}
    n^{2m+1} &= \sum_{r=0}^{m} \coeffA{m}{r} \left[ 1^r \cdot (n-1)^r + 2^r \cdot (n-2)^r + 3^r \cdot (n-3)^r + \cdots + n^r \cdot (n-n)^r  \right] \\
    &= \sum_{r=0}^{m} \coeffA{m}{r} \left[ (n-1)^r + (2n-4)^r + (3n-9)^r + (4n-16)^r + \cdots +  (n^2-n^2)^r  \right]
\end{align*}
For example,
\begin{itemize}
    \item $1^{2m+1} = \sum_{r=0}^{m} \coeffA{m}{r} \left[ 0^r  \right]$
    \item $2^{2m+1} = \sum_{r=0}^{m} \coeffA{m}{r} \left[ 1^r + 0^r  \right]$
    \item $3^{2m+1} = \sum_{r=0}^{m} \coeffA{m}{r} \left[ 2^r + 2^r + 0^r  \right]$
    \item $4^{2m+1} = \sum_{r=0}^{m} \coeffA{m}{r} \left[ 3^r + 4^r + 3^r + 0^r  \right]$
    \item $5^{2m+1} = \sum_{r=0}^{m} \coeffA{m}{r} \left[ 4^r + 6^r + 6^r + 4^r + 0^r  \right]$
    \item $6^{2m+1} = \sum_{r=0}^{m} \coeffA{m}{r} \left[ 5^r + 8^r + 9^r + 8^r + 5^r + 0^r  \right]$
\end{itemize}
Assuming that $x^0=1$ for every $x$.
Interestingly enough that the explicit form above is a Pascal-type identity,
in terms of bivariate function $k(n-k)$ and numbers $\coeffA{m}{r}$,
with similar pattern as Pascal's identity itself~\cite{macmillan2011proofs}.
We may see it by observing the Pascal's identity $(n+1)^{k+1}-1=\sum _{p=0}^{k}{\binom {k+1}{p}}(1^{p}+2^{p}+\dots +n^{p})$.
In particular,
\begin{align*}
    n^{2m+1} - 1
    = \sum_{r=0}^{m} \coeffA{m}{r} \left[ (n-1)^r + (2n-4)^r + (3n-9)^r + (4n-16)^r + \cdots +  (n-1)^r  \right]
\end{align*}
\begin{definition}[Bivariate sum $T_m$]
    For integers $n,k$ and $m \geq 0$
    \label{def:bivariate-sum-Tm}
    \begin{align*}
        T_{m}(n,k) = \sum_{r=0}^{m} \coeffA{m}{r} k^r (n-k)^r
    \end{align*}
\end{definition}

\begin{proposition}[Symmetry of $T_m$]
    \label{prop:Tm-symmetry}
    For integers $n$ and $k$
    \begin{align*}
        T_{m} (n, k) = T_{m} (n, n-k)
    \end{align*}
\end{proposition}

\subsection{Forward decompositions}\label{subsec:forward-decompositions}
\begin{proposition}[Forward Recurrence for $T_m$]
    \label{prop:Tm-recurrence-forward}
    \begin{align*}
        T_{m} (n,k) = \sum_{t=1}^{m+1} (-1)^{t+1} \binom{m+1}{t} T_{m} (n+t, k)
    \end{align*}
    \begin{proof}
        The polynomial $T_{m} (n,k)$ is a polynomial of degree $m$ in $n$.
        Thus, the forward difference with respect to $n$ is
        $\Delta^{m+1} T_{m} (n, k) = \sum_{t=0}^{m+1} (-1)^{t} \binom{m+1}{t} T_{m} (n+t, k) = 0$.
        By isolating $(-1)^{0} \binom{m+1}{0} T_{m} (n-0, k)$ yields
        $T_{m} (n, k) = (-1) \sum_{t=1}^{m+1} (-1)^{t} \binom{m+1}{t} T_{m} (n+t, k)$.
    \end{proof}
\end{proposition}

\begin{proposition}[Odd power forward decomposition]
    \label{prop:odd-power-decomposition-forward}
    For non-negative integers $m$ and $n$
    \begin{align*}
        n^{2m+1} = \sum_{k=1}^{n} \sum_{t=1}^{m+1} (-1)^{t+1} \binom{m+1}{t} T_{m} (n+t, k)
    \end{align*}
    \begin{proof}
        Direct consequence of~\eqref{theorem:odd-power-identity}
        and forward recurrence~\eqref{prop:Tm-recurrence-forward}.
    \end{proof}
\end{proposition}
For example: $3^5 = \binom{3}{1} 1023 - \binom{3}{2} 2643 + \binom{3}{3} 5103$.
Interesting to note that by swapping the signs yields
$(-3)^5 = -\binom{3}{1} 1023 + \binom{3}{2} 2643 - \binom{3}{3} 5103$.

\begin{proposition}[Odd power forward decomposition $m-1$]
    \label{prop:odd-power-decomposition-forward-m-1}
    For non-negative integers $m$ and $n$
    \begin{align*}
        n^{2m-1} = \sum_{k=1}^{n} \sum_{t=1}^{m} (-1)^{t+1} \binom{m}{t} T_{m-1} (n+t, k)
    \end{align*}
    \begin{proof}
        By setting $m \rightarrow m-1$ to~\eqref{prop:odd-power-decomposition-forward}.
    \end{proof}
\end{proposition}

\begin{proposition}[Odd power forward decomposition shifted]
    \label{prop:odd-power-decomposition-forward-shifted}
    For non-negative integers $m$ and $n$
    \begin{align*}
        n^{2m+1} = \sum_{k=0}^{n-1} \sum_{t=1}^{m+1} (-1)^{t+1} \binom{m+1}{t} T_{m} (n+t, k)
    \end{align*}
    \begin{proof}
        Direct consequence of~\eqref{theorem:odd-power-identity},
        forward recurrence~\eqref{prop:Tm-recurrence-forward}, and symmetry~\eqref{prop:Tm-symmetry}.
    \end{proof}
\end{proposition}

\begin{proposition}[Odd power forward decomposition $m-1$ shifted]
    \label{prop:odd-power-decomposition-forward-m-1-shifted}
    For non-negative integers $m$ and $n$
    \begin{align*}
        n^{2m-1} = \sum_{k=0}^{n-1} \sum_{t=1}^{m} (-1)^{t+1} \binom{m}{t} T_{m-1} (n+t, k)
    \end{align*}
    \begin{proof}
        By setting $m \rightarrow m-1$ to~\eqref{prop:odd-power-decomposition-forward} and
        by symmetry~\eqref{prop:Tm-symmetry}.
    \end{proof}
\end{proposition}

\subsection{Forward decompositions multifold}\label{subsec:forward-decompositions-multifold}
\begin{proposition}[Forward Recurrence for $T_m$ multifold]
    \label{prop:Tm-recurrence-forward-multifold}
    \begin{align*}
        T_{m} (n,k) = \sum_{t=1}^{m+s} (-1)^{t+1} \binom{m+s}{t} T_{m} (n+t, k)
    \end{align*}
\end{proposition}

\subsection{Backward decompositions}\label{subsec:backward-decompositions}
\begin{proposition}[Backward Recurrence for $T_m$]
    \label{prop:Tm-recurrence-backward}
    For non-negative integers $m$ and $n$
    \begin{align*}
        T_{m} (n, k) = \sum_{t=1}^{m+1} (-1)^{t+1} \binom{m+1}{t} T_{m} (n-t, k)
    \end{align*}
    \begin{proof}
        The polynomial $T_{m} (n,k)$ is a polynomial of degree $m$ in $n$.
        Thus, the backward difference with respect to $n$ is
        $\nabla^{m+1} T_{m} (n, k) = \sum_{t=0}^{m+1} (-1)^{t} \binom{m+1}{t} T_{m} (n-t, k) = 0$.
        By isolating $(-1)^{0} \binom{m+1}{0} T_{m} (n-0, k)$ yields
        $T_{m} (n, k) = (-1) \sum_{t=1}^{m+1} (-1)^{t} \binom{m+1}{t} T_{m} (n-t, k)$.
    \end{proof}
\end{proposition}

\begin{proposition}[Odd power backward decomposition]
    \label{prop:odd-power-decomposition-backward}
    For non-negative integers $m$ and $n$
    \begin{align*}
        n^{2m+1} = \sum_{k=1}^{n} \sum_{t=1}^{m+1} (-1)^{t+1} \binom{m+1}{t} T_{m} (n-t, k)
    \end{align*}
    \begin{proof}
        Direct consequence of~\eqref{theorem:odd-power-identity}
        and backward recurrence~\eqref{prop:Tm-recurrence-backward}.
    \end{proof}
\end{proposition}

\begin{proposition}[Odd power backward decomposition shifted]
    \label{prop:odd-power-decomposition-backward-shifted}
    For non-negative integers $m$ and $n$
    \begin{align*}
        n^{2m+1} = \sum_{k=0}^{n-1} \sum_{t=1}^{m+1} (-1)^{t+1} \binom{m+1}{t} T_{m} (n-t, k)
    \end{align*}
    \begin{proof}
        Direct consequence of~\eqref{theorem:odd-power-identity},
        backward recurrence~\eqref{prop:Tm-recurrence-backward}, and symmetry~\eqref{prop:Tm-symmetry}.
    \end{proof}
\end{proposition}

\begin{corollary}[Odd power backward decomposition $m-1$]
    \label{cor:odd-power-decomposition-m-1}
    \begin{align*}
        n^{2m-1} = \sum_{k=1}^{n} \sum_{t=1}^{m} (-1)^{t+1} \binom{m}{t} T_{m-1} (n-t, k)
    \end{align*}
    \begin{proof}
        By setting $m \rightarrow m-1$ to~\eqref{prop:odd-power-decomposition-backward}.
    \end{proof}
\end{corollary}

\begin{corollary}[Odd power backward decomposition $m-1$ shifted]
    \label{cor:odd-power-decomposition-m-1-shifted}
    \begin{align*}
        n^{2m-1} = \sum_{k=0}^{n-1} \sum_{t=1}^{m} (-1)^{t+1} \binom{m}{t} T_{m-1} (n-t, k)
    \end{align*}
    \begin{proof}
        By setting $m \rightarrow m-1$ to~\eqref{prop:odd-power-decomposition-backward-shifted}
        and by symmetry~\eqref{prop:Tm-symmetry}.
    \end{proof}
\end{corollary}

\subsection{Backward decompositions multifold}
\label{subsec:backward-decompositions-multifold}
\begin{proposition}[Backward Recurrence for $T_m$ multifold]
    \label{prop:Tm-recurrence-backward-multifold}
    Integer $s \geq 1$
    \begin{align*}
        T_{m} (n, k) = \sum_{t=1}^{m+s} (-1)^{t+1} \binom{m+s}{t} T_{m} (n-t, k)
    \end{align*}
\end{proposition}

\begin{proposition}[Odd power backward decomposition multifold]
    \label{prop:odd-power-decomposition-backward-multifold}
    For non-negative integers $m$, $n$ and $s \geq 1$
    \begin{align*}
        n^{2m+1} = \sum_{k=1}^{n} \sum_{t=1}^{m+1} (-1)^{t+1} \binom{m+s}{t} T_{m} (n-t, k)
    \end{align*}
    \begin{proof}
        Direct consequence of~\eqref{theorem:odd-power-identity}
        and backward recurrence~\eqref{prop:Tm-recurrence-backward-multifold}.
    \end{proof}
\end{proposition}

\begin{proposition}[Odd power backward decomposition shifted multifold]
    \label{prop:odd-power-decomposition-backward-shifted-multifold}
    For non-negative integers $m$, $n$ and $s \geq 1$
    \begin{align*}
        n^{2m+1} = \sum_{k=0}^{n-1} \sum_{t=1}^{m+1} (-1)^{t+1} \binom{m+s}{t} T_{m} (n-t, k)
    \end{align*}
    \begin{proof}
        Direct consequence of~\eqref{theorem:odd-power-identity},
        backward recurrence~\eqref{prop:Tm-recurrence-backward-multifold},
        and symmetry~\eqref{prop:Tm-symmetry}.
    \end{proof}
\end{proposition}

\begin{corollary}[Odd power backward decomposition $m-1$ multifold]
    \label{cor:odd-power-decomposition-m-1-multifold}
    For non-negative integers $m$, $n$ and $s \geq 0$
    \begin{align*}
        n^{2m-1} = \sum_{k=1}^{n} \sum_{t=1}^{m} (-1)^{t+1} \binom{m+s}{t} T_{m-1} (n-t, k)
    \end{align*}
    \begin{proof}
        By setting $m \rightarrow m-1$ to~\eqref{prop:odd-power-decomposition-backward-multifold}.
    \end{proof}
\end{corollary}

\begin{corollary}[Odd power backward decomposition $m-1$ shifted multifold]
    \label{cor:odd-power-decomposition-m-1-shifted-multifold}
    For non-negative integers $m$, $n$ and $s \geq 0$
    \begin{align*}
        n^{2m-1} = \sum_{k=0}^{n-1} \sum_{t=1}^{m} (-1)^{t+1} \binom{m+s}{t} T_{m-1} (n-t, k)
    \end{align*}
    \begin{proof}
        By setting $m \rightarrow m-1$ to~\eqref{prop:odd-power-decomposition-backward-shifted-multifold}
        and by symmetry~\eqref{prop:Tm-symmetry}.
    \end{proof}
\end{corollary}

\subsection{Binomial forms}\label{subsec:binomial-forms}

\begin{corollary}[Binomial form]
    \label{prop:binomial-form}
    For integers $n$ and $a$ such that $n+2a \geq 0$
    \begin{align*}
    (n+2a)^{2m+1} = \sum_{r=0}^{m} \sum_{k=-a+1}^{n+a} \coeffA{m}{r} (k+a)^r (n+a-k)^r
    \end{align*}
\end{corollary}

\begin{corollary}[Shifted binomial form]
    \label{prop:shifted-binomial-form}
    For integers $n$ and $a$ such that $n+2a \geq 0$
    \begin{align*}
    (n+2a)^{2m+1} = \sum_{r=0}^{m} \sum_{k=-a}^{n+a-1} \coeffA{m}{r} (k+a)^r (n+a-k)^r
    \end{align*}
\end{corollary}

\begin{corollary}[Centered binomial form]
    \label{cor:centered-binomial-form}
    For integers $n$ and $a$ such that $n+a \geq 0$
    \begin{align*}
    (n+a)^{2m+1} = \sum_{r=0}^{m} \sum_{k=-\frac{a}{2}+1}^{n+\frac{a}{2}} \coeffA{m}{r} \left( k+\frac{a}{2} \right)^r \left( n+\frac{a}{2}-k \right)^r
    \end{align*}
\end{corollary}

\begin{corollary}[Shifted centered binomial form]
    \label{cor:shifted-centered-binomial-form}
    For integers $n$ and $a$ such that $n-2a \geq 0$
    \begin{align*}
    (n+a)^{2m+1} = \sum_{r=0}^{m} \sum_{k=-\frac{a}{2}}^{n+\frac{a}{2}-1} \coeffA{m}{r} \left( k+\frac{a}{2} \right)^r \left( n+\frac{a}{2}-k \right)^r
    \end{align*}
\end{corollary}

\begin{proposition}[Negated binomial form]
    \label{prop:negated-binomial-form}
    For integers $n$ and $a$ such that $n-2a \geq 0$
    \begin{align*}
    (n-2a)^{2m+1} = \sum_{r=0}^{m} \sum_{k=a+1}^{n-a} \coeffA{m}{r} (k-a)^r (n-a-k)^r
    \end{align*}
    \begin{proof}
        By observing the summation limits we can see that $k$ runs as $k=a+1,a+2,a+3,\ldots,a+n-a$, which
        implies that $(k-a)=1,2,3,\ldots, n$.
        By observing the term $(n-k-a)$ we see that $(n-k-a)=n-1,n-2,n-3,\ldots,0$.
        Thus, by reindexing the sum
        $(n-2a)^{2m+1} = \sum_{r=0}^{m} \sum_{k=1}^{n-2a} \coeffA{m}{r} (a+k-a)^r (n-(a+k)-a)^r$
        the statement~\eqref{prop:odd-power-binomial} is equivalent to~\eqref{theorem:odd-power-identity}
        with setting $n \rightarrow n-2a$.
    \end{proof}
\end{proposition}

\begin{corollary}[Shifted negated binomial form]
    \label{prop:shifted-negated-binomial-form}
    For integers $n$ and $a$ such that $n-2a \geq 0$
    \begin{align*}
    (n-2a)^{2m+1} = \sum_{r=0}^{m} \sum_{k=a}^{n-a-1} \coeffA{m}{r} (k-a)^r (n-a-k)^r
    \end{align*}
\end{corollary}

\begin{corollary}[Centered negated binomial form]
    \label{cor:centered-negated-binomial-form}
    For integers $n$ and $a$ such that $n-a \geq 0$
    \begin{align*}
    (n-a)^{2m+1} = \sum_{r=0}^{m} \sum_{k=\frac{a}{2}}^{n-\frac{a}{2}-1} \coeffA{m}{r} \left( k-\frac{a}{2} \right)^r \left(n-\frac{a}{2}-k \right)^r
    \end{align*}
\end{corollary}

\begin{corollary}[Shifted centered negated binomial form]
    \label{cor:shifted-centered-negated-binomial-form}
    For integers $n$ and $a$ such that $n-a \geq 0$
    \begin{align*}
    (n-a)^{2m+1} = \sum_{r=0}^{m} \sum_{k=\frac{a}{2}+1}^{n-\frac{a}{2}} \coeffA{m}{r} \left( k-\frac{a}{2} \right)^r \left(n-\frac{a}{2}-k \right)^r
    \end{align*}
\end{corollary}

\subsection{Sums of powers}\label{subsec:sums-of-powers}
\begin{proposition}[Sums of odd powers]
    \label{prop:sum-of-odd-powers}
    \begin{align*}
        \sum_{n=1}^{p} n^{2m+1} = \sum_{n=1}^{p} \sum_{k=1}^{n} \sum_{r=0}^{m} \coeffA{m}{r} k^r (n-k)^r
    \end{align*}
\end{proposition}

\begin{proposition}[Sums of odd powers forward decomposition]
    \label{prop:sum-odd-power-decomposition-forward}
    For non-negative integers $m$ and $n$
    \begin{align*}
        \sum_{n=1}^{p} n^{2m+1} = \sum_{n=1}^{p} \sum_{k=1}^{n} \sum_{t=1}^{m+1} (-1)^{t+1} \binom{m+1}{t} T_{m} (n+t, k)
    \end{align*}
\end{proposition}

\begin{proposition}[Sum of odd powers backward decomposition]
    \label{prop:sum-odd-power-decomposition-backward}
    For non-negative integers $m$ and $n$
    \begin{align*}
        \sum_{n=1}^{p} n^{2m+1} = \sum_{n=1}^{p} \sum_{k=1}^{n} \sum_{t=1}^{m+1} (-1)^{t+1} \binom{m+1}{t} T_{m} (n-t, k)
    \end{align*}
\end{proposition}

\subsection{Double bivariate identities}\label{subsec:double-bivariate-identities}


