Thus, the conjecture~\eqref{conj:odd-power-identity} is true

\begin{theorem}[Odd power identity]
    \label{theorem:odd-power-identity}
    There is a set of coefficients $\coeffA{m}{0}, \coeffA{m}{1}, \ldots, \coeffA{m}{m}$ such that
    \begin{align*}
        n^{2m+1} = \sum_{r=0}^{m} \sum_{k=1}^{n} \coeffA{m}{r} k^r (n-k)^r
    \end{align*}
\end{theorem}

\begin{definition}[Bivariate sum $T_m$]
    For integers $n,k$ and $m \geq 0$
    \label{def:bivariate-sum-Tm}
    \begin{align*}
        T_{m}(n,k) = \sum_{r=0}^{m} \coeffA{m}{r} k^r (n-k)^r
    \end{align*}
\end{definition}

\begin{proposition}[Symmetry of $T_m$]
    \label{prop:Tm-symmetry}
    For integers $n$ and $k$
    \begin{align*}
        T_{m} (n, k) = T_{m} (n, n-k)
    \end{align*}
\end{proposition}

\begin{proposition}[Backward Recurrence for $T_m$]
    \label{prop:Tm-recurrence-backward}
    For non-negative integers $m$ and $n$
    \begin{align*}
        T_{m} (n, k) = \sum_{t=1}^{m+1} (-1)^{t+1} \binom{m+1}{t} T_{m} (n-t, k)
    \end{align*}
    \begin{proof}
        The polynomial $T_{m} (n,k)$ is a polynomial of degree $m$ in $n$.
        Thus, the backward difference with respect to $n$ is
        $\nabla^{m+1} T_{m} (n, k) = \sum_{t=0}^{m+1} (-1)^{t} \binom{m+1}{t} T_{m} (n-t, k) = 0$.
        By isolating $(-1)^{0} \binom{m+1}{0} T_{m} (n-0, k)$ yields
        $T_{m} (n, k) = (-1) \sum_{t=1}^{m+1} (-1)^{t} \binom{m+1}{t} T_{m} (n-t, k)$.
    \end{proof}
\end{proposition}

\begin{proposition}[Odd power backward decomposition]
    \label{prop:odd-power-decomposition-backward}
    For non-negative integers $m$ and $n$
    \begin{align*}
        n^{2m+1} = \sum_{k=1}^{n} \sum_{t=1}^{m+1} (-1)^{t+1} \binom{m+1}{t} T_{m} (n-t, k)
    \end{align*}
    \begin{proof}
        Direct consequence of~\eqref{theorem:odd-power-identity}
        and backward recurrence~\eqref{prop:Tm-recurrence-backward}.
    \end{proof}
\end{proposition}

\begin{proposition}[Odd power backward decomposition shifted]
    \label{prop:odd-power-decomposition-backward-shifted}
    For non-negative integers $m$ and $n$
    \begin{align*}
        n^{2m+1} = \sum_{k=0}^{n-1} \sum_{t=1}^{m+1} (-1)^{t+1} \binom{m+1}{t} T_{m} (n-t, k)
    \end{align*}
    \begin{proof}
        Direct consequence of~\eqref{theorem:odd-power-identity},
        backward recurrence~\eqref{prop:Tm-recurrence-backward}, and symmetry~\eqref{prop:Tm-symmetry}.
    \end{proof}
\end{proposition}

\begin{corollary}[Odd power backward decomposition $m-1$]
    \label{cor:odd-power-decomposition-m-1}
    \begin{align*}
        n^{2m-1} = \sum_{k=1}^{n} \sum_{t=1}^{m} (-1)^{t+1} \binom{m}{t} T_{m-1} (n-t, k)
    \end{align*}
    \begin{proof}
        By setting $m \rightarrow m-1$ to~\eqref{prop:odd-power-decomposition-backward}.
    \end{proof}
\end{corollary}

\begin{proposition}[Forward Recurrence for $T_m$]
    \label{prop:Tm-recurrence-forward}
    \begin{align*}
        T_{m} (n,k) = \sum_{t=1}^{m+1} (-1)^{t+1} \binom{m+1}{t} T_{m} (n+t, k)
    \end{align*}
    \begin{proof}
        The polynomial $T_{m} (n,k)$ is a polynomial of degree $m$ in $n$.
        Thus, the forward difference with respect to $n$ is
        $\Delta^{m+1} T_{m} (n, k) = \sum_{t=0}^{m+1} (-1)^{t} \binom{m+1}{t} T_{m} (n+t, k) = 0$.
        By isolating $(-1)^{0} \binom{m+1}{0} T_{m} (n-0, k)$ yields
        $T_{m} (n, k) = (-1) \sum_{t=1}^{m+1} (-1)^{t} \binom{m+1}{t} T_{m} (n+t, k)$.
    \end{proof}
\end{proposition}

\begin{proposition}[Odd power forward decomposition]
    \label{prop:odd-power-decomposition-forward}
    For non-negative integers $m$ and $n$
    \begin{align*}
        n^{2m+1} = \sum_{k=1}^{n} \sum_{t=1}^{m+1} (-1)^{t+1} \binom{m+1}{t} T_{m} (n+t, k)
    \end{align*}
    \begin{proof}
        Direct consequence of~\eqref{theorem:odd-power-identity}
        and forward recurrence~\eqref{prop:Tm-recurrence-forward}.
    \end{proof}
\end{proposition}

\begin{proposition}[Odd power forward decomposition shifted]
    \label{prop:odd-power-decomposition-forward-shifted}
    For non-negative integers $m$ and $n$
    \begin{align*}
        n^{2m+1} = \sum_{k=0}^{n-1} \sum_{t=1}^{m+1} (-1)^{t+1} \binom{m+1}{t} T_{m} (n+t, k)
    \end{align*}
    \begin{proof}
        Direct consequence of~\eqref{theorem:odd-power-identity},
        forward recurrence~\eqref{prop:Tm-recurrence-forward}, and symmetry~\eqref{prop:Tm-symmetry}.
    \end{proof}
\end{proposition}

\begin{proposition}[Central Recurrence for $T_m$]
    \label{prop:Tm-recurrence-central}
    \begin{align*}
        T_{m} \left(n + \frac{m}{2}, k \right)
        = \sum_{t=1}^{m+1} (-1)^{t+1} \binom{m+1}{t} T_{m} \left( n+\frac{m}{2}-t, k \right)
    \end{align*}
    \begin{proof}
        The polynomial $T_{m} (n,k)$ is a polynomial of degree $m$ in $n$.
        Thus, the central difference with respect to $n$ is
        $\delta^{m+1} T_{m} (n, k) = \sum_{t=0}^{m+1} (-1)^{t} \binom{m+1}{t} T_{m} \left(n+\frac{m}{2} - t, k\right) = 0$.
        By isolating $(-1)^{0} \binom{m+1}{0} T_{m} \left(n+\frac{m}{2}-0, k\right)$ yields
        $T_{m} \left(n+\frac{m}{2}-0, k\right) = (-1) \sum_{t=1}^{m+1} (-1)^{t} \binom{m+1}{t} T_{m} (n+t, k)$.
    \end{proof}
\end{proposition}

\begin{proposition}[Odd power central decomposition]
    For non-negative integers $m$ and $n$
    \label{prop:odd-power-decomposition-central}
    \begin{align*}
    (n+m)
        ^{4m+1} = \sum_{k=1}^{n} \sum_{t=1}^{2m+1} (-1)^{t} \binom{2m+1}{t} T_{2m} (n+m-t, 2)
    \end{align*}
    \begin{proof}
        Direct consequence of~\eqref{theorem:odd-power-identity}, central recurrence~\eqref{prop:Tm-recurrence-central},
        by setting $m \rightarrow 2m$.
    \end{proof}
\end{proposition}

\begin{proposition}[Odd power central decomposition shifted]
    For non-negative integers $m$ and $n$
    \label{prop:odd-power-decomposition-central-shifted}
    \begin{align*}
    (n+m)
        ^{4m+1} = \sum_{k=0}^{n-1} \sum_{t=1}^{2m+1} (-1)^{t} \binom{2m+1}{t} T_{2m} (n+m-t, 2)
    \end{align*}
    \begin{proof}
        Direct consequence of~\eqref{theorem:odd-power-identity},
        central recurrence~\eqref{prop:Tm-recurrence-central}, and symmetry~\eqref{prop:Tm-symmetry},
        by setting $m \rightarrow 2m$.
    \end{proof}
\end{proposition}

\begin{proposition}[Odd power binomial form]
    \label{prop:odd-power-binomial}
    For integers $n$ and $a$ such that $n-2a \geq 0$
    \begin{align*}
    (n-2a)
        ^{2m+1} = \sum_{r=0}^{m} \sum_{k=a+1}^{n-a} \coeffA{m}{r} (k-a)^r (n-k-a)^r
    \end{align*}
    \begin{proof}
        By observing the summation limits we can see that $k$ runs as $k=a+1,a+2,a+3,\ldots,a+n-a$, which
        implies that $(k-a)=1,2,3,\ldots, n$.
        By observing the term $(n-k-a)$ we see that $(n-k-a)=n-1,n-2,n-3,\ldots,0$.
        Thus, by reindexing the sum
        $(n-2a)^{2m+1} = \sum_{r=0}^{m} \sum_{k=1}^{n-2a} \coeffA{m}{r} (a+k-a)^r (n-(a+k)-a)^r$
        the statement~\eqref{prop:odd-power-binomial} is equivalent to~\eqref{theorem:odd-power-identity}
        with setting $n \rightarrow n-2a$.
    \end{proof}
\end{proposition}

\begin{corollary}[Odd power binomial form shifted]
    \label{prop:odd-power-binomial-shifted}
    For integers $n$ and $a$ such that $n-2a \geq 0$
    \begin{align*}
    (n-2a)
        ^{2m+1} = \sum_{r=0}^{m} \sum_{k=a}^{n-a-1} \coeffA{m}{r} (k-a)^r (n-k-a)^r
    \end{align*}
\end{corollary}

\begin{proposition}[Sum of odd powers]
    \label{prop:sum-of-odd-powers}
    \begin{align*}
        \sum_{t=1}^{n} t^{2m+1} = \sum_{t=1}^{n} \sum_{k=1}^{t} \sum_{r=0}^{m} \coeffA{m}{r} k^r (n-k)^r
    \end{align*}
\end{proposition}

