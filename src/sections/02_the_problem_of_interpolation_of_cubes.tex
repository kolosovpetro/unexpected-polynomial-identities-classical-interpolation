Once there was a curious student, curious in general and especially in mathematics.
He was not an educated mathematician, but he was strongly captured by the mathematical
beauty and aesthetics.

One day this student was having fun with tables of finite differences, precisely finite differences of cubes.
By observing the table
\begin{table}[H]
    \begin{center}
        \setlength\extrarowheight{-6pt}
        \begin{tabular}{c|cccc}
            $n$ & $n^3$ & $\Delta(n^3)$ & $\Delta^2(n^3)$ & $\Delta^3(n^3)$ \\
            \hline
            0   & 0     & 1             & 6               & 6               \\
            1   & 1     & 7             & 12              & 6               \\
            2   & 8     & 19            & 18              & 6               \\
            3   & 27    & 37            & 24              & 6               \\
            4   & 64    & 61            & 30              & 6               \\
            5   & 125   & 91            & 36              &                 \\
            6   & 216   & 127           &                 &                 \\
            7   & 343   &               &                 &
        \end{tabular}
    \end{center}
    \caption{Table of finite differences of the polynomial $n^3$.} \label{tab:table}
\end{table}

The first question that triggered his mind was
\begin{question}
    \label{quest:interpolation-cubes}
    How to reconstruct the value of $n^3$ from its finite differences?
\end{question}
Precisely, the inquiry was to find a way to reconstruct the values of $\{0, 1, 8, 27, 64, \ldots\}$
having values of finite differences in the table.

In its essence, the problem is so old that it can be traced back to ancient Babylonian and Greek times,
several centuries BC and first centuries AD~\cite{gautschi2012interpolation}.
The process of finding new data points based on the range of a discrete set
of known data points is called interpolation.
Interpolation, as we know it today, was developed in 1674--1684 by Isaac Newton
in his works referenced as foundation of classical interpolation theory~\cite{meijering2002chronology}.
For instance, Newton's series for $n^3$ is
\begin{align*}
    n^3 = 6 \binom{n}{3} + 6\binom{n}{2} + 1 \binom{n}{1} + 0\binom{n}{0}
\end{align*}
because $f(x) = \sum_{k=0}^{d} \Delta^{d-k} f(0) \binom{x}{d-k}$, see~\cite[~p. 190]{graham1994concrete}.

Great!
But there is one thing, the student who has risen the question~\eqref{quest:interpolation-cubes}
had no clue about interpolation theory at all.
What he decided then?
Exactly, he decided to try to re-invent interpolation formula himself,
fueled by the purest feeling of mystery.
His mind was occupied by only a single thought:
\textit{
    All mathematical truths exist timelessly, we only reveal and describe them.
}
That mindset inspired our student to start his own mathematical journey.

By observing the table of finite differences~\eqref{tab:finite-differences-cubes} we can notice that
the first order finite difference of cubes may be expressed in terms of its
third order finite difference $\Delta^3(n^3) = 6$, as follows
\begin{align*}
    \begin{split}
        \Delta(0^3) &= 1+6 \cdot 0 \\
        \Delta(1^3) &= 1+6\cdot0+6\cdot1 \\
        \Delta(2^3) &= 1+6\cdot0+6\cdot1+6\cdot2 \\
        \Delta(3^3) &= 1+6\cdot0+6\cdot1+6\cdot2+6\cdot3 \\
        &\; \; \vdots \\
        \Delta(n^3) &= 1+6\cdot0+6\cdot1+6\cdot2+6\cdot3 + \cdots + 6n
    \end{split}
\end{align*}
By using sigma notation, we get
\begin{align*}
    \Delta(n^3) = 1+6\cdot0+6\cdot1+6\cdot2+6\cdot3+\cdots+6\cdot n = 1 + 6 \sum_{k=0}^{n} k
\end{align*}

However, there is a more beautiful way to prove that $\Delta(n^3) = 1 + 6 \sum_{k=0}^{n} k$.
We refer to one of the finest articles in the area of polynomials and power sums,
that is \textit{Johann Faulhaber and sums of powers} written by Donald Knuth~\cite{knuth1993johann}.
Indeed, this article is a great source to reach piece of mind in mathematics.
We now focus on the following odd power identities shown at~\cite[~p. 9]{knuth1993johann}
\begin{align*}
    n^1 &= \binom{n}{1} \\
    n^3 &= 6 \binom{n+1}{3} + \binom{n}{1} \\
    n^5 &= 120 \binom{n+2}{5} + 30 \binom{n+1}{3} + \binom{n}{1}
\end{align*}

It is in particular interesting that well-known identity which connects triangular numbers $\binom{n+1}{2}$
and finite differences of cubes becomes clear and obvious
\begin{align*}
    \Delta n^3
    = (n+1)^3 - n^3
    =  6 \binom{n+1}{2} + \binom{n}{0}
\end{align*}
It easy to see that
\begin{align*}
    \Delta n^3
    = \left[ 6 \binom{n+2}{3} + \binom{n+1}{1} \right] - \left[ 6 \binom{n+1}{3} + \binom{n}{1} \right]
    = 6 \binom{n+1}{2} + \binom{n}{0}
\end{align*}
because $\binom{n}{k} = \binom{n-1}{k} + \binom{n-1}{k-1}$.

Moreover, the concept above allows to reach $N$-fold power sums $\sum^N k^{2m+1}$
or finite differences $\Delta^N k^{2m+1}$ of odd powers by simply altering
binomial coefficients indexes.
Quite strong and impressive.

We can observe that triangular numbers $\binom{n+1}{2}$ are equivalent to
\begin{align*}
    \binom{n+1}{2} = \sum_{k=0}^{n} k
\end{align*}
because $\binom{n+1}{m+1} = \sum_{k=0}^{n} \binom{k}{m}$.
This leads to the identity in finite differences of cubes
\begin{align*}
    \Delta n^3 = (n+1)^3 - n^3 = 1 + 6 \sum_{k=0}^{n} k
\end{align*}
An experienced mathematician would immediately notice a spot to apply Faulhaber's formula~\cite{beardon1996sums}
to expand the term $\sum_{k=0}^{n} k$
\begin{align*}
    \sum_{k=0}^{n} k = \frac{1}{2}(n+n^2)
\end{align*}
Thus, the finite difference $\Delta(n^3)$ takes a well-known form such
that matches Binomial theorem~\cite{abramowitz1988handbook}
\begin{align*}
    \Delta(n^3)
    = 1 + 6 \left[ \frac{1}{2}(n+n^2) \right]
    = 1 + 3 n + 3 n^2
    = \sum_{k=0}^{2} \binom{3}{k} n^k
\end{align*}
And\ldots that could be the end of the story, isn't it?
Because all what remains is to say that
\begin{align*}
    n^3
    = \sum_{k=0}^{n-1} (k+1)^3 - k^3
    = \sum_{k=0}^{n-1} \left( 1 + 6 \sum_{t=0}^{k} t \right)
    = \sum_{k=0}^{n-1} 1 + 3 k + 3 k^2
\end{align*}
Thus, the polynomial $n^3$ is interpolated successfully, and thus, the question~\eqref{quest:interpolation-cubes} is answered
positively.
Because we have successfully found the function that matches $n^3$ from the values of its finite differences from the
table~\eqref{tab:finite-differences-cubes}.

However, not this time.
Luckily enough (say), the student who has stated the question~\eqref{quest:interpolation-cubes}
wasn't really aware of the approaches above neither.
What a lazy student!
Probably, that's exactly the case when unawareness leads to a fresh sight to century-old questions,
leading to unexpected results and new insights.
Instead, our student spotted a little bit different pattern in $\Delta n^3= 6 \binom{n+1}{2} + \binom{n}{0}$.

Consider the polynomial $n^3$ as sum of its finite differences
\begin{align*}
    n^3
    &= [1+6\cdot0] \\
    &+ [1+6\cdot0+6\cdot1] \\
    &+ [1+6\cdot0+6\cdot1+6\cdot2] + \cdots \\
    &+ [1+6\cdot0+6\cdot1+6\cdot2+\cdots+6\cdot(n-1)]
\end{align*}
We can observe that the term $1$ appears $n$ times, the item $6\cdot0$ appears $n-0$ times,
the item $6\cdot1$ appears $n-1$ times and so on.
By rearranging recurring common terms
\begin{align*}
    n^3 = n
    &+ [(n-0) \cdot 6 \cdot 0] \\
    &+ [(n-1)\cdot6\cdot1] \\
    &+ [(n-2)\cdot6\cdot2] + \cdots \\
    &+ [(n-k)\cdot 6 \cdot k] + \cdots \\
    &+ [1\cdot6\cdot(n-1)]
\end{align*}
By applying compact sigma sum notation yields an identity for cubes $n^3$
\begin{align*}
    n^3 = n + \sum_{k=0}^{n-1} 6k(n-k)
\end{align*}
We can freely move the term $n$ under the summation because there are exactly $n$ iterations.
Therefore,
\begin{align*}
    n^3 = \sum_{k=0}^{n-1} 6k(n-k) + 1
\end{align*}
By inspecting the expression $6k(n-k) + 1$, we can notice that it is symmetric over $k$.
Let be $T(n,k) = 6k(n-k) + 1$ then
\begin{align*}
    T(n,k) = T(n,n-k)
\end{align*}
This symmetry allows us to alter summation bounds easily.
Hence,
\begin{align*}
    n^3 = \sum_{k=1}^{n} 6k(n-k) + 1
\end{align*}
By arranging the values of $T(n,k)$ as a triangular array, we see that cube identities indeed are true
\begin{table}[H]
    \setlength\extrarowheight{-6pt}
    \begin{tabular}{c|cccccccc}
        $n/k$ & 0 & 1  & 2  & 3  & 4  & 5  & 6  & 7 \\
        \hline
        0     & 1 &    &    &    &    &    &    &   \\
        1     & 1 & 1  &    &    &    &    &    &   \\
        2     & 1 & 7  & 1  &    &    &    &    &   \\
        3     & 1 & 13 & 13 & 1  &    &    &    &   \\
        4     & 1 & 19 & 25 & 19 & 1  &    &    &   \\
        5     & 1 & 25 & 37 & 37 & 25 & 1  &    &   \\
        6     & 1 & 31 & 49 & 55 & 49 & 31 & 1  &   \\
        7     & 1 & 37 & 61 & 73 & 73 & 61 & 37 & 1
    \end{tabular}
    \caption{Values of $T(n,k) = 6k(n-k) + 1$.
    See the sequence \href{https://oeis.org/A287326}{\texttt{A287326}} in OEIS
    ~\cite{oeis_numerical_triangle_row_sums_give_cubes}.}
    \label{tab:triangle_row_sums_give_cubes}
\end{table}

The following recurrence holds for $T(n,k)$
\begin{align*}
    T(n, k) = 2T(n-1, k) - T(n-2, k)
\end{align*}
Finally, our curious student has reached the first milestone, by finding his own
answer to the question~\eqref{quest:interpolation-cubes} and the answer was positive.
What an excitement he felt!
However, it wouldn't take long.
Indeed, curiosity is not something that can be fulfilled completely,
and thus new questions arise.
Somehow, he got a strong feeling that something bigger, something even more general
hides behind the identity $n^3 = \sum_{k=1}^{n} 6k(n-k) + 1$.
Quite intuitive he was.
