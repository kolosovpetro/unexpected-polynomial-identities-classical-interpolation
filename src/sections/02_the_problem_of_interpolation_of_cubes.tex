Back then, in 2016 being a student at the faculty of mechanical engineering,
I remember myself playing with finite differences of the polynomial $n^3$ over the domain of natural numbers $n\in\mathbb{N}$
having at most $0 \leq n \leq 20$ values.
Looking to the values in my finite difference tables, the first and very naive question that came to my mind was
\begin{question}
    Is it possible to re-assemble the value of the polynomial $n^3$ backwards
    having its finite differences?
\end{question}
The answer to this question is certainly \textit{Yes}, by utilizing interpolation methods.
Interpolation is a process of finding new data points based on the range of a discrete set of known data points.
It has been well-developed in between 1674--1684
by Issac Newton's fundamental works, nowadays known as foundation of classical interpolation
theory~\cite{meijering2002chronology}.

At that time, in 2016, I was a first-year mechanical engineering undergraduate.
Therefore, due to lack of knowledge in mathematics, I started re-inventing interpolation
formula myself, fueled by purest passion and feeling of mystery.
\textit{All mathematical laws and relations exist from the very beginning, but we only reveal and describe them}, I thought.
That mindset truly inspired me so that my own mathematical journey began.

Let's start considering the table of finite differences of the polynomial $n^3$
\begin{table}[H]
    \begin{center}
        \setlength\extrarowheight{-6pt}
        \begin{tabular}{c|cccc}
            $n$ & $n^3$ & $\Delta(n^3)$ & $\Delta^2(n^3)$ & $\Delta^3(n^3)$ \\
            \hline
            0   & 0     & 1             & 6               & 6               \\
            1   & 1     & 7             & 12              & 6               \\
            2   & 8     & 19            & 18              & 6               \\
            3   & 27    & 37            & 24              & 6               \\
            4   & 64    & 61            & 30              & 6               \\
            5   & 125   & 91            & 36              &                 \\
            6   & 216   & 127           &                 &                 \\
            7   & 343   &               &                 &
        \end{tabular}
    \end{center}
    \caption{Table of finite differences of the polynomial $n^3$.} \label{tab:table}
\end{table}

First and foremost, we can observe that finite difference $\Delta(n^3)$ of the polynomial $n^3$
can be expressed through summation over $n$, e.g
\begin{align}
    \label{eq:cubes_interpolation}
    \begin{split}
        \Delta(0^3) &= 1+6 \cdot 0 \\
        \Delta(1^3) &= 1+6\cdot0+6\cdot1 \\
        \Delta(2^3) &= 1+6\cdot0+6\cdot1+6\cdot2 \\
        \Delta(3^3) &= 1+6\cdot0+6\cdot1+6\cdot2+6\cdot3 \\
        &\; \; \vdots
    \end{split}
\end{align}
Finally reaching its generic form
\begin{equation}
    \Delta(n^3) = 1+6\cdot0+6\cdot1+6\cdot2+6\cdot3+\cdots+6\cdot n = 1 + 6 \sum_{k=0}^{n} k\label{eq:general-cube-eq}
\end{equation}
The one experienced mathematician would immediately notice a spot to apply Faulhaber's formula~\cite{beardon1996sums}
to expand the term $\sum_{k=0}^{n} k$ reaching expected result that matches Binomial theorem~\cite{abramowitz1988handbook},
so that
\begin{equation*}
    \sum_{k=0}^{n} k = \frac{1}{2}(n+n^2)
\end{equation*}
Then our relation~\eqref{eq:general-cube-eq} immediately turns into Binomial expansion
\begin{equation}
    \Delta(n^3) = (n+1)^3 - n^3 = 1 + 6 \left[ \frac{1}{2}(n+n^2) \right] = 1 + 3 n + 3 n^2 = \sum_{k=0}^{2} \binom{3}{k} n^k
    \label{eq:cubes-difference-binomial-theorem}
\end{equation}
However, as was said, I was not the experienced one mathematician back then,
so that I reviewed the relation~\eqref{eq:general-cube-eq} from a little bit different perspective.
Not following the convenient solution~\eqref{eq:cubes-difference-binomial-theorem},
I have introduced the explicit formula for cubes, using~\eqref{eq:cubes_interpolation}
\begin{align}
    \label{eq:rearrangement_to_get_cubes}
    n^3 &= [1+6\cdot0]+[1+6\cdot0+6\cdot1]+[1+6\cdot0+6\cdot1+6\cdot2]+\cdots \nonumber \\
    &+[1+6\cdot0+6\cdot1+6\cdot2+\cdots+6\cdot(n-1)]
\end{align}
Then, rearranging the terms in equation~\eqref{eq:rearrangement_to_get_cubes} so that it turns into summation
in terms of $k (n-k)$
\begin{equation*}
    \begin{split}
        n^3 &= n + [(n-0) \cdot 6 \cdot 0] + [(n-1)\cdot6\cdot1] + [(n-2)\cdot6\cdot2] + \cdots \\
        &\cdots + [(n-k)\cdot 6 \cdot k] + \cdots + [1\cdot6\cdot(n-1)]
    \end{split}
\end{equation*}
By applying compact sigma notation and moving $n$ under summation because there is exactly $n$ iteration, yields
\begin{equation}
    \label{eq:cube_identity}
    n^3 = n + \sum_{k=1}^{n} 6k(n-k); \quad \quad n^3 = \sum_{k=1}^{n} 6k(n-k) + 1
\end{equation}
\begin{table}[H]
    \setlength\extrarowheight{-6pt}
    \begin{tabular}{c|cccccccc}
        $n/k$ & 0 & 1  & 2  & 3  & 4  & 5  & 6  & 7 \\
        \hline
        0     & 1 &    &    &    &    &    &    &   \\
        1     & 1 & 1  &    &    &    &    &    &   \\
        2     & 1 & 7  & 1  &    &    &    &    &   \\
        3     & 1 & 13 & 13 & 1  &    &    &    &   \\
        4     & 1 & 19 & 25 & 19 & 1  &    &    &   \\
        5     & 1 & 25 & 37 & 37 & 25 & 1  &    &   \\
        6     & 1 & 31 & 49 & 55 & 49 & 31 & 1  &   \\
        7     & 1 & 37 & 61 & 73 & 73 & 61 & 37 & 1
    \end{tabular}
    \caption{Values of $T(n,k) = 6k(n-k) + 1$.
    See the sequence \href{https://oeis.org/A287326}{\texttt{A287326}} in OEIS
    ~\cite{oeis_numerical_triangle_row_sums_give_cubes}.}
    \label{tab:triangle_row_sums_give_cubes}
\end{table}

Therefore, we have reached our base case by successfully interpolating the polynomial $n^3$.
Fairly enough that the next curiosity would be
\begin{question}
    Well, if the relation~\eqref{eq:cube_identity} true for the polynomial $n^3$,
    then is it true that~\eqref{eq:cube_identity} can be generalized for higher powers, e.g.\ for $n^4$ or $n^5$ similarly?
    \label{question:higher_powers}
\end{question}
That was the next question, however without any expectation of the final form of generalized formula.
Long story short, the answer to this question is also \textit{Yes}, by utilizing certain approaches
in terms of systems of linear equations or recurrence formula, which is discussed.

Let us begin from the background and history overview of systems of linear equations approach.
%In 2018, Albert Tkaczyk published two papers~\cite{tkaczyk2018problem, tkaczyk2018continuation}
presenting analogous identities for polynomials $n^5, \; n^7$ and $n^9$
derived in a manner similar to $n^3 = \sum_{k=1}^{n} 6k(n-k) + 1$.
Tkaczyk assumed that the identity for $n^5$ takes the following explicit form
\begin{align*}
    n^5 = \sum_{k=1}^{n} \left[ A k^2(n-k)^2 + Bk(n-k) + C \right]
\end{align*}
where $A,B,C$ are yet-unknown coefficients.
We denote $A,B,C$ as $\coeffA{2}{0}, \coeffA{2}{1}, \coeffA{2}{2}$
to reach the compact form of double sum
\begin{align*}
    n^5 = \sum_{k=1}^{n} \sum_{r=0}^{2} \coeffA{2}{r} k^r (n-k)^r
\end{align*}
By observing the equation above, the potential form of generalized odd-power identity becomes more obvious.
One important note to add here, we define $0^x = 1$ for all $x$, see~\cite[~p. 162]{graham1994concrete}.
This is because when $k=n$ and $r=0$ the term $k^r (n-k)^r = n^0 \cdot 0^0$, thus we must define $0^x = 1$
for all $x$.

To evaluate the set of coefficients $\coeffA{2}{0}, \coeffA{2}{1}, \coeffA{2}{2}$
we construct and solve a certain system of linear equations, which is
built as follows
\begin{align*}
    n^5 = \coeffA{2}{0} \sum_{k=1}^{n} k^0 (n-k)^0 + \coeffA{2}{1} \sum_{k=1}^{n} k^1 (n-k)^1 + \coeffA{2}{2} \sum_{k=1}^{n} k^2 (n-k)^2
\end{align*}
By expanding the sums $\sum_{k=1}^{n} k^r (n-k)^r$ using Faulhaber's formula~\cite{beardon1996sums}, we get
an equation
\begin{equation*}
    \coeffA{2}{0} n
    + \coeffA{2}{1} \left[ \frac{1}{6} (n^3-n) \right]
    + \coeffA{2}{2} \left[ \frac{1}{30} (n^5-n) \right] - n^5 = 0
\end{equation*}
By multiplying by $30$ both right-hand side and left-hand side, we get
\begin{equation*}
    30 \coeffA{2}{0} n + 5 \coeffA{2}{1} (n^3-n) + \coeffA{2}{2} (n^5-n) - 30n^5 = 0
\end{equation*}
By expanding the brackets and rearranging the terms
\begin{equation*}
    30 \coeffA{2}{0} - 5 \coeffA{2}{1} n + 5 \coeffA{2}{1} n^3 - \coeffA{2}{2} n + \coeffA{2}{2} n^5 - 30n^5 = 0
\end{equation*}
By combining the common terms, we obtain
\begin{equation*}
    n (30 \coeffA{2}{0} - 5 \coeffA{2}{1} - \coeffA{2}{2}) + 5 \coeffA{2}{1} n^3 + n^5 (\coeffA{2}{2} - 30) = 0
\end{equation*}
Therefore,
\begin{equation*}
    \begin{cases}
        30 \coeffA{2}{0} - 5 \coeffA{2}{1} - \coeffA{2}{2} &= 0 \\
        \coeffA{2}{1} &= 0 \\
        \coeffA{2}{2} - 30 &= 0
    \end{cases}
\end{equation*}
By solving the system above, we evaluate the coefficients $\coeffA{2}{0}, \coeffA{2}{1}, \coeffA{2}{2}$
\begin{equation*}
    \begin{cases}
        \coeffA{2}{2} &= 30 \\
        \coeffA{2}{1} &= 0 \\
        \coeffA{2}{0} &= 1
    \end{cases}
\end{equation*}
Thus, the identity for $n^5$
\begin{equation*}
    n^5 = \sum_{k=1}^{n} 30k^2(n-k)^2 + 1
\end{equation*}
Again, the terms $30k^2(n-k)^2 + 1$ are symmetric over $k$.
Let be $T_2 (n,k) = 30k^2(n-k)^2 + 1$ then
\begin{align*}
    T_2 (n,k) = T_2 (n,n-k)
\end{align*}
By arranging the values of $T_{2} (n,k)$ as a triangular array, we see that the identity for $n^5$ is indeed true
\begin{table}[H]
    \setlength\extrarowheight{-6pt}
    \begin{tabular}{c|cccccccc}
        $n/k$ & 0 & 1    & 2    & 3    & 4    & 5    & 6    & 7 \\
        \hline
        0     & 1 &      &      &      &      &      &      &   \\
        1     & 1 & 1    &      &      &      &      &      &   \\
        2     & 1 & 31   & 1    &      &      &      &      &   \\
        3     & 1 & 121  & 121  & 1    &      &      &      &   \\
        4     & 1 & 271  & 481  & 271  & 1    &      &      &   \\
        5     & 1 & 481  & 1081 & 1081 & 481  & 1    &      &   \\
        6     & 1 & 751  & 1921 & 2431 & 1921 & 751  & 1    &   \\
        7     & 1 & 1081 & 3001 & 4321 & 4321 & 3001 & 1081 & 1
    \end{tabular}
    \caption{Values of $30k^2(n-k)^2 + 1$.
    See the OEIS entry \href{https://oeis.org/A300656}{\texttt{A300656}}
    ~\cite{oeis_numerical_triangle_row_sums_give_fifth_powers}.}
    \label{tab:row-sums-gives-fifth-power}
\end{table}

The following recurrence holds for $T_2 (n,k)$
\begin{align*}
    T_2 (n, k) = 3T_2(n-1, k) - 3T_2(n-2, k) + T_2(n-3, k)
\end{align*}
Which is indeed true because
\begin{align*}
    T_2 (6,2) = 3 \cdot 1081 - 3 \cdot 481 + 271 = 1921
\end{align*}

%Consider an example for $n^7$.
%\begin{align*}
%    n^7 =
%    \coeffA{3}{0} n
%    + \coeffA{3}{1} \left[ \frac{1}{6} (n^3-n) \right]
%    &+ \coeffA{3}{2} \left[ \frac{1}{30} (n^5-n) \right] \\
%    &+ \coeffA{3}{3} \left[ \frac{1}{420} (3n^7+7n^3-10n) \right]
%\end{align*}
%By multiplying by $30$ both right-hand side and left-hand side, we get
%\begin{align*}
%    420 \coeffA{3}{0} n + 70 \coeffA{2}{1} (n^3-n)
%    &+ 14 \coeffA{2}{2} (n^5-n) \\
%    &+ \coeffA{3}{3} (3n^7+7n^3-10n) - 420n^7 = 0
%\end{align*}
%By expanding the brackets and rearranging terms, we obtain
%\begin{align*}
%    420 \coeffA{3}{0} n
%    &- 70 \coeffA{3}{1} + 70 \coeffA{3}{1} n^3 - 14 \coeffA{3}{2} n + 14 \coeffA{3}{2} n^5 \\
%    &- 10 \coeffA{3}{3} n + 7 \coeffA{3}{3} n^3 + 3 \coeffA{3}{3} n^7 - 420n^7 = 0
%\end{align*}
%Combining the common terms yields
%\begin{align*}
%    n (420 \coeffA{3}{0} - 70 \coeffA{3}{1} - 14 \coeffA{3}{2} - 10 \coeffA{3}{3})
%    &+ n^3 (70 \coeffA{3}{1} + 7 \coeffA{3}{3}) \\
%    &+ n^5 14 \coeffA{3}{2}
%    + n^7 (3 \coeffA{3}{3} - 420)
%    = 0
%\end{align*}
%Therefore, we receive the following system of linear equations
%\begin{align*}
%    \begin{cases}
%        420 \coeffA{3}{0} - 70 \coeffA{3}{1} - 14 \coeffA{3}{2} - 10 \coeffA{3}{3} &= 0 \\
%        70 \coeffA{3}{1} + 7 \coeffA{3}{3} &= 0 \\
%        \coeffA{3}{2} - 30 &= 0 \\
%        3 \coeffA{3}{3} - 420 &= 0
%    \end{cases}
%\end{align*}
%By solving it for $\coeffA{3}{0}, \coeffA{3}{1}, \coeffA{3}{2}, \coeffA{3}{3}$, we obtain
%\begin{equation*}
%    \begin{cases}
%        \coeffA{3}{3} &= 140 \\
%        \coeffA{3}{2} &= 0 \\
%        \coeffA{3}{1} = -\frac{7}{70} \coeffA{3}{3} &= -14 \\
%        \coeffA{3}{0} = \frac{(70 \coeffA{3}{1} + 10 \coeffA{3}{3})}{420} &= 1
%    \end{cases}
%\end{equation*}
%Therefore, the identity for $n^7$ holds
%\begin{equation*}
%    n^7 = \sum_{k=1}^{n} 140 k^3 (n-k)^3 - 14k(n-k) + 1
%\end{equation*}
%The values of $140 k^3 (n-k)^3 - 14k(n-k) + 1$ are symmetric over $k$.
%Let be $T_3 (n,k) = 140 k^3 (n-k)^3 - 14k(n-k) + 1$ then
%\begin{align*}
%    T_3 (n,k) = T_3 (n, n-k)
%\end{align*}
%By arranging the values of $T_{3} (n,k)$ as a triangular array, we see that the identity for $n^7$ is indeed true
%\begin{table}[H]
    \setlength\extrarowheight{-6pt}
    \begin{tabular}{c|cccccccc}
        $n/k$ & 0 & 1     & 2      & 3      & 4      & 5      & 6     & 7 \\
        \hline
        0     & 1 &       &        &        &        &        &       &   \\
        1     & 1 & 1     &        &        &        &        &       &   \\
        2     & 1 & 127   & 1      &        &        &        &       &   \\
        3     & 1 & 1093  & 1093   & 1      &        &        &       &   \\
        4     & 1 & 3739  & 8905   & 3739   & 1      &        &       &   \\
        5     & 1 & 8905  & 30157  & 30157  & 8905   & 1      &       &   \\
        6     & 1 & 17431 & 71569  & 101935 & 71569  & 17431  & 1     &   \\
        7     & 1 & 30157 & 139861 & 241753 & 241753 & 139861 & 30157 & 1
    \end{tabular}
    \caption{Values of $140 k^3 (n-k)^3 - 14k(n-k) + 1$.
    See the OEIS entry \href{https://oeis.org/A300785}{\texttt{A300785}}
    ~\cite{oeis_numerical_triangle_row_sums_give_seventh_powers}.}
    \label{tab:row-sums-gives-seventh-power}
\end{table}

%The following recurrence holds for $T_{3} (n,k)$
%\begin{align*}
%    T_{3} (n, k) = 4T_{3} (n-1, k) - 6T_{3} (n-2, k) + 4T_{3} (n-3, k) - T_{3} (n-4, k)
%\end{align*}
%Which is true indeed because
%\begin{align*}
%    T_{3} (7, 1) = 4 \cdot 17431 -6 \cdot 8905 + 4 \cdot 3739 - 1 \cdot 1093 = 30157
%\end{align*}

%
%In 2018, the recurrence relation~\cite{alekseyev2018mathoverflow} that evaluates the coefficients $\coeffA{m}{r}$ for
non-negative integer $m$ was provided by Max Alekseyev, George Washington University.
The main idea of Alekseyev's approach was to utilize a recurrence relation to evaluate the set of coefficients $\coeffA{m}{r}$
starting from the base case $\coeffA{m}{m}$ and then evaluating the next coefficient $\coeffA{m}{m-1}$
recursively, and so on up to $\coeffA{m}{0}$.
We utilize Binomial theorem $(n-k)^r=\sum_{t=0}^{r} (-1)^t \binom{r}{t} n^{r-t} k^t$ and a specific version
of Faulhaber's formula~\cite{beardon1996sums}
\begin{align*}
    \sum_{k=1}^{n} k^{p}
    = \frac{1}{p+1}\sum_{j=0}^{p} \binom{p+1}{j} \bernoulli{j} n^{p+1-j}
    &= \frac{1}{p+1} \left[ \sum_{j=0}^{p+1} \binom{p+1}{j} \bernoulli{j} n^{p+1-j} \right] - \frac{\bernoulli{p+1}}{p+1} \\
    &= \frac{1}{p+1} \left[ \sum_{j} \binom{p+1}{j} \bernoulli{j} n^{p+1-j} \right] - \frac{\bernoulli{p+1}}{p+1}
\end{align*}
The reason we use modified version of Faulhaber's formula is because we want to omit summation bounds, for simplicity.
This helps us to collapse the common terms across complex sums, see also~\cite[~p. 2]{knuth1992two}.
Therefore, we expand the sum $\sum_{k=1}^{n} k^{r} (n-k)^{r}$ using Binomial theorem
\begin{align*}
    \sum_{k=1}^{n} k^{r} (n-k)^{r} = \sum_{t=0}^{r} (-1)^t \binom{r}{t} n^{r-t} \sum_{k=1}^{n} k^{t+r}
\end{align*}
By applying Faulhaber's formula above, we obtain
\begin{align*}
    \sum_{k=1}^{n} k^{r} (n-k)^{r}
    = \sum_{t=0}^{r} (-1)^t \binom{r}{t} n^{r-t} \left[ \left( \frac{1}{t+r+1} \sum_{j} \binom{t+r+1}{j} \bernoulli{j} n^{t+r+1-j} \right) - \frac{\bernoulli{t+r+1}}{t+r+1} \right]
\end{align*}
By moving the common term $\frac{(-1)^t}{t+r+1}$ out of brackets
\begin{align*}
    \sum_{k=1}^{n} k^{r} (n-k)^{r}
    = \sum_{t=0}^{r} \binom{r}{t} \frac{(-1)^t}{t+r+1} \left[ \sum_{j} \binom{t+r+1}{j} \bernoulli{j} n^{2r+1-j} - \bernoulli{t+r+1} n^{r-t} \right]
\end{align*}
By expanding the brackets
\begin{align*}
    \sum_{k=1}^{n} k^{r} (n-k)^{r}
    &= \left[ \sum_{t=0}^{r} \binom{r}{t} \frac{(-1)^t}{t+r+1} \sum_{j} \binom{t+r+1}{j} \bernoulli{j} n^{2r+1-j}  \right] \\
    &- \left[ \sum_{t=0}^{r} \binom{r}{t} \frac{(-1)^t}{t+r+1} \bernoulli{t+r+1} n^{r-t} \right]
\end{align*}
By moving the sum in $j$ and omitting summation bounds in $t$
\begin{align*}
    \sum_{k=1}^{n} k^{r} (n-k)^{r}
    = \left[ \sum_{j, t} \binom{r}{t} \frac{(-1)^t}{t+r+1} \binom{t+r+1}{j} \bernoulli{j} n^{2r+1-j}  \right]
    - \left[ \sum_{t} \binom{r}{t} \frac{(-1)^t}{t+r+1} \bernoulli{t+r+1} n^{r-t} \right]
\end{align*}
By rearranging the sums we obtain
\begin{align}
    \label{eq:rearranging-terms}
    \sum_{k=1}^{n} k^{r} (n-k)^{r}
    &= \left[ \sum_{j} \bernoulli{j} n^{2r+1-j} \sum_{t} \binom{r}{t} \frac{(-1)^t}{t+r+1} \binom{t+r+1}{j}  \right] \\
    &- \left[ \sum_{t} \binom{r}{t} \frac{(-1)^t}{t+r+1} \bernoulli{t+r+1} n^{r-t} \right] \nonumber
\end{align}
We can notice that
\begin{lemma}
    \label{cor:combinatorial-identity}
    For integers $r, j$
    \begin{align*}
        \sum_{t} \binom{r}{t} \frac{(-1)^t}{r+t+1} \binom{r+t+1}{j}
        =\begin{cases}
             \frac{1}{(2r+1) \binom{2r}r} & \text{if } j=0\\
             \frac{(-1)^r}{j} \binom{r}{2r-j+1} & \text{if } j>0
        \end{cases}
    \end{align*}
    \begin{proof}
        An elegant proof is done by Markus Scheuer in~\cite{scheuer2023mathstackexchange}.
    \end{proof}
\end{lemma}
In particular, the sum in lemma~\eqref{cor:combinatorial-identity} is zero for $0< j \leq r$.
To utilize the lemma~\eqref{cor:combinatorial-identity}, we have to move $j=0$ out of summation
in~\eqref{eq:rearranging-terms} to avoid division by zero in $\frac{(-1)^r}{j}$.
Therefore,
\begin{equation*}
    \begin{split}
        \sum_{k=1}^{n} k^{r} (n-k)^{r}
        &= \frac{1}{(2r+1) \binom{2r}r} n^{2r+1}
        + \left[ \sum_{j = 1}^{\infty} \bernoulli{j} n^{2r+1-j} \sum_{t} \binom{r}{t} \frac{(-1)^t}{t+r+1} \binom{t+r+1}{j} \right] \\
        &- \left[ \sum_{t} \binom{r}{t} \frac{(-1)^t}{t+r+1} \bernoulli{t+r+1} n^{r-t} \right]
    \end{split}
\end{equation*}
Hence, we simplify the equation~\eqref{eq:rearranging-terms} by using lemma~\eqref{cor:combinatorial-identity}
\begin{equation*}
    \begin{split}
        \sum_{k=1}^{n} k^{r} (n-k)^{r}
        &= \frac{1}{(2r+1) \binom{2r}r} n^{2r+1}
        + \underbrace{\left[ \sum_{j = 1}^{\infty} \frac{(-1)^r}{j} \binom{r}{2r-j+1} \bernoulli{j} n^{2r-j+1} \right]}_{(\star)} \\
        &- \underbrace{\left[ \sum_{t} \binom{r}{t} \frac{(-1)^t}{t+r+1} \bernoulli{t+r+1} n^{r-t} \right]}_{(\diamond)}
    \end{split}
\end{equation*}
By introducing $\ell=2r-j+1$ to $(\star)$ and $\ell=r-t$ to $(\diamond)$
we collapse the common terms across two sums
\begin{equation*}
    \begin{split}
        \sum_{k=1}^{n} k^{r} (n-k)^{r}
        &= \frac{1}{(2r+1) \binom{2r}r} n^{2r+1}
        + \left[ \sum_{\ell} \frac{(-1)^r}{2r+1-\ell} \binom{r}{\ell} \bernoulli{2r+1-\ell} n^{\ell} \right] \\
        &- \left[ \sum_{\ell} \binom{r}{\ell} \frac{(-1)^{r-\ell}}{2r+1-\ell} \bernoulli{2r+1-\ell} n^{\ell} \right]\\
        &= \frac{1}{(2r+1) \binom{2r}r} n^{2r+1} + 2 \sum_{\mathrm{odd \; \ell}} \frac{(-1)^r}{2r+1-\ell} \binom{r}{\ell} \bernoulli{2r+1-\ell} n^{\ell}
    \end{split}
\end{equation*}
Assuming that $\coeffA{m}{r}$ is defined by the odd-power identity in~\eqref{conj:odd-power-identity},
we obtain the following relation for polynomials in $n$
\begin{equation*}
    \sum_{r=0}^{m} \coeffA{m}{r} \frac{1}{(2r+1) \binom{2r}r} n^{2r+1}
    + 2 \sum_{r=0}^{m} \sum_{\mathrm{odd \; \ell}} \coeffA{m}{r} \frac{(-1)^r}{2r+1-\ell} \binom{r}{\ell} \bernoulli{2r+1-\ell} n^{\ell}
    \equiv n^{2m+1}
\end{equation*}
Replacing odd $\ell$ by $\ell = 2k+1$ we get
\begin{align*}
    \sum_{r=0}^{m} \coeffA{m}{r} \frac{1}{(2r+1) \binom{2r}{r}} n^{2r+1} + 2 \sum_{r=0}^{m} \sum_{k=0}^{\infty} \coeffA{m}{r} \frac{(-1)^r}{2r-2k} \binom{r}{2k+1} \bernoulli{2r-2k} n^{2k+1}  \equiv n^{2m+1}
\end{align*}
By simplifying the term $2$
\begin{align}
    \label{eq:main_relation}
    \sum_{r=0}^{m} \coeffA{m}{r} \frac{1}{(2r+1) \binom{2r}{r}} n^{2r+1} + \sum_{r=0}^{m} \sum_{k=0}^{\infty} \coeffA{m}{r} \frac{(-1)^r}{r-k} \binom{r}{2k+1} \bernoulli{2r-2k} n^{2k+1}  \equiv n^{2m+1}
\end{align}
Basically, the relation~\eqref{eq:main_relation} is the generating function we utilize to
evaluate the values of $\coeffA{m}{0}, \coeffA{m}{1}, \ldots, \coeffA{m}{m}$.
We now fix the unused values of $\coeffA{m}{r}$ so that $\coeffA{m}{r} = 0$ for every $r < 0$ or $r > m$.

Taking the coefficient of $n^{2m+1}$ in~\eqref{eq:main_relation} yields
\begin{align*}
    \coeffA{m}{m} = (2m+1)\binom{2m}{m}
\end{align*}
because $\coeffA{m}{m} \frac{1}{(2m+1) \binom{2m}{m}} = 1$.

That's may not be immediately clear why the coefficient of $n^{2m+1}$ is $(2m+1)\binom{2m}{m}$.
To extract the coefficient of $n^{2m+1}$ from the expression~\eqref{eq:main_relation},
we isolate the relevant terms by setting $r = m$ in the first sum,
and $k = m$ in the second sum.
This gives
\begin{align*}
[n^{2m+1}]
    &\left(
         \sum_{r=0}^{m} \coeffA{m}{r} \frac{1}{(2r+1) \binom{2r}{r}} n^{2r+1}
         + \sum_{r=0}^{m} \sum_{k=0}^{\infty} \coeffA{m}{r} \frac{(-1)^r}{r-k} \binom{r}{2k+1} \bernoulli{2r - 2k} n^{2k+1}
         - n^{2m+1}
    \right) \\
    &= \coeffA{m}{m} \frac{1}{(2m+1) \binom{2m}{m}}
    + \sum_{r=0}^{m} \coeffA{m}{r} \frac{(-1)^r}{r - m} \binom{r}{2m+1} \bernoulli{2r - 2m}
    - 1
\end{align*}
We observe that the sum
\begin{align*}
    \sum_{r=0}^{m} \coeffA{m}{r} \frac{(-1)^r}{r - m} \binom{r}{2m+1} \bernoulli{2r - 2m}
\end{align*}
does not contribute to the determination of the coefficients because the binomial coefficient
$\binom{r}{2m+1}$ vanishes for all $r \leq m$.
Consequently, all terms in the sum are zero.
Thus,
\begin{align*}
    \coeffA{m}{m} \frac{1}{(2m+1) \binom{2m}{m}}  - 1 = 0 \implies \coeffA{m}{m} = (2m+1) \binom{2m}{m}
\end{align*}

Taking the coefficient of $n^{2d+1}$ for an integer $d$ in the range $\frac{m}{2} \leq d \leq m-1$ in~\eqref{eq:main_relation} gives
\begin{align*}
[n^{2d+1}]
    &\left( \sum_{r=0}^{m} \coeffA{m}{r} \frac{1}{(2r+1) \binom{2r}{r}} n^{2r+1} + \sum_{r=0}^{m} \sum_{k=0}^{\infty} \coeffA{m}{r} \frac{(-1)^r}{r-k} \binom{r}{2k+1} \bernoulli{2r-2k} n^{2k+1} - n^{2m+1} \right) \\
    &= \coeffA{m}{d} \frac{1}{(2d+1) \binom{2d}{d}} + \sum_{r=0}^m \coeffA{m}{r} \frac{(-1)^r}{r-d} \binom{r}{2d+1} \bernoulli{2r-2d}.
\end{align*}
For every $\frac{m}{2} \leq d$, the binomial coefficient $\binom{r}{2d+1}$ vanishes, because for all $r \leq m$
holds $r < 2d+1$.
As a particular example, when $r = m$ and $d = \frac{m}{2}$, we have
\begin{align*}
    \binom{m}{m+1} = 0.
\end{align*}
Therefore, the entire sum involving $\binom{r}{2d+1}$ vanishes, and we conclude
\begin{align*}
    \coeffA{m}{d} \frac{1}{(2d+1) \binom{2d}{d}} = 0 \implies \coeffA{m}{d} = 0.
\end{align*}
Hence, for all integers $d$ such that $\frac{m}{2} \leq d \leq m-1$, the coefficient $\coeffA{m}{d} = 0$.
In contrast, for values $d \leq \frac{m}{2} - 1$, the binomial coefficient $\binom{r}{2d+1}$ can be nonzero; for instance, if $r = m$ and $d = \frac{m}{2} - 1$, then
\begin{align*}
    \binom{m}{m - 1} \neq 0,
\end{align*}
allowing the corresponding terms to contribute to the determination of $\coeffA{m}{d}$.

Taking the coefficient of $n^{2d+1}$ for $d$ in the range $\frac{m}{4} \leq d < \frac{m}{2}$ in~\eqref{eq:main_relation}, we obtain
\begin{align*}
    \coeffA{m}{d} \frac{1}{(2d+1) \binom{2d}{d}}
    + 2 (2m+1) \binom{2m}{m} \binom{m}{2d+1} \frac{(-1)^m}{2m - 2d} \bernoulli{2m - 2d} = 0.
\end{align*}
Solving for $\coeffA{m}{d}$ yields
\begin{equation*}
    \coeffA{m}{d}
    = (-1)^{m-1} \frac{(2m+1)!}{d! \, d! \, m! \, (m - 2d - 1)!} \cdot \frac{1}{m - d} \bernoulli{2m - 2d}.
\end{equation*}

Proceeding recursively, we can compute each coefficient $\coeffA{m}{r}$ for integers $r$ in the ranges
\begin{align*}
    \frac{m}{2^{s+1}} \leq r < \frac{m}{2^s}, \quad \text{for } s = 1, 2, \ldots
\end{align*}
by using previously computed values $\coeffA{m}{d}$ for $d > r$, via the relation
\begin{equation*}
    \coeffA{m}{r} =
    (2r+1) \binom{2r}{r} \sum_{d = 2r+1}^{m}
    \coeffA{m}{d} \binom{d}{2r+1} \frac{(-1)^{d-1}}{d - r} \bernoulli{2d - 2r}.
\end{equation*}

Finally, we define the following recurrence relation for coefficients $\coeffA{m}{r}$
\begin{proposition}
    For integers $m$ and $r$
    \label{prop:coefficients_a}
    \begin{align*}
        \coeffA{m}{r} =
        \begin{cases}
        (2r+1)
            \binom{2r}{r} & \mathrm{if} \; r=m \\
            (2r+1) \binom{2r}{r} \sum_{d = 2r+1}^{m} \coeffA{m}{d} \binom{d}{2r+1} \frac{(-1)^{d-1}}{d-r}
            \bernoulli{2d-2r} & \mathrm{if} \; 0 \leq r<m \\
            0 & \mathrm{if} \; r<0 \; \mathrm{or} \; r>m
        \end{cases}
    \end{align*}
    where $\bernoulli{t}$ are Bernoulli numbers~\cite{bateman1953higher}.
    It is assumed that $\bernoulli{1}=\frac{1}{2}$.
\end{proposition}

Thus, the conjecture~\eqref{conj:odd-power-identity} is true

\begin{theorem}
    \label{theorem:odd-power-identity}
    There is a set of coefficients $\coeffA{m}{0}, \coeffA{m}{1}, \ldots, \coeffA{m}{m}$ such that
    \begin{align*}
        n^{2m+1} = \sum_{r=0}^{m} \sum_{k=1}^{n} \coeffA{m}{r} k^r (n-k)^r
    \end{align*}
\end{theorem}

For example,
\begin{table}[H]
    \begin{center}
        \setlength\extrarowheight{-6pt}
        \begin{tabular}{c|cccccccc}
            $m/r$ & 0 & 1       & 2      & 3      & 4   & 5    & 6     & 7 \\ [3px]
            \hline
            0     & 1 &         &        &        &     &      &       &       \\
            1     & 1 & 6       &        &        &     &      &       &       \\
            2     & 1 & 0       & 30     &        &     &      &       &       \\
            3     & 1 & -14     & 0      & 140    &     &      &       &       \\
            4     & 1 & -120    & 0      & 0      & 630 &      &       &       \\
            5     & 1 & -1386   & 660    & 0      & 0   & 2772 &       &       \\
            6     & 1 & -21840  & 18018  & 0      & 0   & 0    & 12012 &       \\
            7     & 1 & -450054 & 491400 & -60060 & 0   & 0    & 0     & 51480
        \end{tabular}
    \end{center}
    \caption{Coefficients $\coeffA{m}{r}$. See OEIS sequences
    ~\cite{oeis_numerators_of_the_coefficient_a_m_r,oeis_denominators_of_the_coefficient_a_m_r}.}
    \label{tab:table_of_coefficients_a}
\end{table}

Properties of the coefficients $\coeffA{m}{r}$
\begin{itemize}
    \item $\coeffA{m}{m} = \binom{2m}{m}$
    \item $\coeffA{m}{r} = 0$ for $m < 0$ and $r > m$
    \item $\coeffA{m}{r} = 0$ for $r < 0$
    \item $\coeffA{m}{r} = 0$ for $\frac{m}{2} \leq r < m$
    \item $\coeffA{m}{0} = 1$ for $m \geq 0$
    \item $\coeffA{m}{r}$ are integers for $m \leq 11$
    \item Row sums: $\sum_{r=0}^{m} \coeffA{m}{r} = 2^{2m+1} - 1$
\end{itemize}

\begin{align*}
    n^3 &= \sum_{k=1}^{n} 6k(n-k) + 1 \\
    n^5 &= \sum_{k=1}^{n} 30k^2(n-k)^2 + 1 \\
    n^7 &= \sum_{k=1}^{n} 140 k^3 (n-k)^3 - 14k(n-k) + 1 \\
    n^9 &= \sum_{k=1}^{n} 630 k^4(n-k)^4 - 120k(n-k) + 1 \\
    n^{11} &= \sum_{k=1}^{n} 2772 k^5(n-k)^5 + 660 k^2(n-k)^2 - 1386k(n-k) + 1 \\
    n^{13} &= \sum_{k=1}^{n} 51480 k^7(n-k)^7 - 60060 k^3(n-k)^3 + 491400k^2(n-k)^{2} - 450054k(n-k) + 1
\end{align*}

\begin{proposition}
    \begin{align*}
        T_{m} (n, k) = \sum_{t=1}^{m+1} (-1)^t \binom{m+1}{t} T_{m} (n-t, k)
    \end{align*}
\end{proposition}

\begin{proposition}
    \begin{align*}
        n^{2m+1} = \sum_{k=1}^{n} \sum_{t=1}^{m+1} (-1)^t \binom{m+1}{t} T_{m} (n-t, k)
    \end{align*}
\end{proposition}

\begin{proposition}
    \begin{align*}
        n^{2m-1} = \sum_{k=1}^{n} \sum_{t=1}^{m} (-1)^t \binom{m}{t} T_{m-1} (n-t, k)
    \end{align*}
\end{proposition}

