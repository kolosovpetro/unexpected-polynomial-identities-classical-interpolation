Once there was a curious student, curious in general and especially in mathematics.
He was not an educated mathematician, but he was strongly captured by the mathematical
beauty and aesthetics.

One day this student was having fun with tables of finite differences, precisely finite differences of cubes.
By observing the table
\begin{table}[H]
    \begin{center}
        \setlength\extrarowheight{-6pt}
        \begin{tabular}{c|cccc}
            $n$ & $n^3$ & $\Delta(n^3)$ & $\Delta^2(n^3)$ & $\Delta^3(n^3)$ \\
            \hline
            0   & 0     & 1             & 6               & 6               \\
            1   & 1     & 7             & 12              & 6               \\
            2   & 8     & 19            & 18              & 6               \\
            3   & 27    & 37            & 24              & 6               \\
            4   & 64    & 61            & 30              & 6               \\
            5   & 125   & 91            & 36              &                 \\
            6   & 216   & 127           &                 &                 \\
            7   & 343   &               &                 &
        \end{tabular}
    \end{center}
    \caption{Table of finite differences of the polynomial $n^3$.} \label{tab:table}
\end{table}

The first question that triggered his mind was
\begin{question}
    How to reconstruct the value of $n^3$ from its finite differences?
\end{question}
The question is quite old, it can be traced back to ancient Babylonian and Greek times,
several centuries BC and first centuries AD [citation].
The process is called interpolation which is a process
of finding new data points based on the range of a discrete set
of known data points.
Interpolation we know nowadays was developed in 1674--1684 by Issac Newton
in his works referenced as foundation of classical interpolation theory [citation].

Great!
But there is one thing, the one who rised the question (1.1) had no clue about all this.
What he decided then?
Exactly, he decided to try to re-invent interpolation formula himself,
fueled by the purest feeling of mystery.
His mind was occupied by only a single thought:
\textit{
    All mathematical truths exist timelessly, we only reveal and describe them.
}
That mindset inspired our student to start his own mathematical journey.

By observing the table of finite differences (1) we can notice that
the first order finite difference of cubes may be expressed in terms of its
third order finite difference $\Delta^3(n^3) = 6$, as follows
\begin{align}
    \label{eq:cubes_interpolation}
    \begin{split}
        \Delta(0^3) &= 1+6 \cdot 0 \\
        \Delta(1^3) &= 1+6\cdot0+6\cdot1 \\
        \Delta(2^3) &= 1+6\cdot0+6\cdot1+6\cdot2 \\
        \Delta(3^3) &= 1+6\cdot0+6\cdot1+6\cdot2+6\cdot3 \\
        &\; \; \vdots \\
        \Delta(n^3) &= 1+6\cdot0+6\cdot1+6\cdot2+6\cdot3 + \cdots + 6n
    \end{split}
\end{align}
By using sigma notation for sums we get
\begin{equation}
    \Delta(n^3) = 1+6\cdot0+6\cdot1+6\cdot2+6\cdot3+\cdots+6\cdot n = 1 + 6 \sum_{k=0}^{n} k
    \label{eq:general-cube-eq}
\end{equation}
An experienced mathematician would immediately notice a spot to apply Faulhaber's formula~\cite{beardon1996sums}
to expand the term $\sum_{k=0}^{n} k$
%so that
\begin{equation*}
    \sum_{k=0}^{n} k = \frac{1}{2}(n+n^2)
\end{equation*}
Thus, the identity $\Delta(n^3) = 1 + 6 \sum_{k=0}^{n} k$ takes a well-known view
that matches Binomial theorem~\cite{abramowitz1988handbook}
so that
\begin{align}
    \Delta(n^3) = (n+1)^3 - n^3 = 1 + 6 \left[ \frac{1}{2}(n+n^2) \right] = 1 + 3 n + 3 n^2 = \sum_{k=0}^{2} \binom{3}{k} n^k
    \label{eq:cubes-difference-binomial-theorem}
\end{align}

There is a more beautiful way to express finite difference of cubes $\Delta(n^3)$.
We refer to one of the prominent articles in area of polynomials, power sums etc.,
that is \textit{Johann Faulhaber and sums of powers} written by Donald Knuth~\cite{knuth1993johann}.
Indeed, this article is a great source to reach piece of mind in mathematics.
The thing that occupied my attention was the odd power identity in terms of Binomial coefficients and
Central factorial numbers, which can be found on page 9
\begin{align*}
    n^1 &= \binom{n}{1} \\
    n^3 &= 6 \binom{n+1}{3} + \binom{n}{1} \\
    n^5 &= 120 \binom{n+2}{5} + 30 \binom{n+1}{3} + \binom{n}{1}
\end{align*}

It is particularly interesting that well-known identity in terms of triangular numbers [citation]
and finite differences of cubes becomes clear and obvious
\begin{align*}
    \Delta n^3 = (n+1)^3 - n^3
    =  6 \binom{n+1}{2} + \binom{n}{0}
\end{align*}
where $\binom{n+1}{2}$ are triangular numbers.
It is true that
\begin{align*}
    \Delta n^3 = \left[ 6 \binom{n+2}{3} + \binom{n+1}{1} \right] - \left[ 6 \binom{n+1}{3} + \binom{n}{1} \right] = 6 \binom{n+1}{2} + \binom{n}{0}
\end{align*}
by means of binomial coefficients' recurrence.
Moreover, the concept above allows to reach $N$-order power sum $\sum^N k^{2m+1}$
or finite difference $\Delta^N k^{2m+1}$ of odd powers simply by changing
binomial coefficients indexes.
Quite strong and impressive.
%However, as was said, I was not the experienced one mathematician back then,
%so that I reviewed the relation~\eqref{eq:general-cube-eq} from a little bit different perspective.
%Not following the convenient solution~\eqref{eq:cubes-difference-binomial-theorem},
%I have introduced the explicit formula for cubes, using~\eqref{eq:cubes_interpolation}
%\begin{align}
%    \label{eq:rearrangement_to_get_cubes}
%    n^3 &= [1+6\cdot0]+[1+6\cdot0+6\cdot1]+[1+6\cdot0+6\cdot1+6\cdot2]+\cdots \nonumber \\
%    &+[1+6\cdot0+6\cdot1+6\cdot2+\cdots+6\cdot(n-1)]
%\end{align}
%Then, rearranging the terms in equation~\eqref{eq:rearrangement_to_get_cubes} so that it turns into summation
%in terms of $k (n-k)$
%\begin{equation*}
%    \begin{split}
%        n^3 &= n + [(n-0) \cdot 6 \cdot 0] + [(n-1)\cdot6\cdot1] + [(n-2)\cdot6\cdot2] + \cdots \\
%        &\cdots + [(n-k)\cdot 6 \cdot k] + \cdots + [1\cdot6\cdot(n-1)]
%    \end{split}
%\end{equation*}
%By applying compact sigma notation and moving $n$ under summation because there is exactly $n$ iteration, yields
%\begin{equation}
%    \label{eq:cube_identity}
%    n^3 = n + \sum_{k=1}^{n} 6k(n-k); \quad \quad n^3 = \sum_{k=1}^{n} 6k(n-k) + 1
%\end{equation}
%\begin{table}[H]
    \setlength\extrarowheight{-6pt}
    \begin{tabular}{c|cccccccc}
        $n/k$ & 0 & 1  & 2  & 3  & 4  & 5  & 6  & 7 \\
        \hline
        0     & 1 &    &    &    &    &    &    &   \\
        1     & 1 & 1  &    &    &    &    &    &   \\
        2     & 1 & 7  & 1  &    &    &    &    &   \\
        3     & 1 & 13 & 13 & 1  &    &    &    &   \\
        4     & 1 & 19 & 25 & 19 & 1  &    &    &   \\
        5     & 1 & 25 & 37 & 37 & 25 & 1  &    &   \\
        6     & 1 & 31 & 49 & 55 & 49 & 31 & 1  &   \\
        7     & 1 & 37 & 61 & 73 & 73 & 61 & 37 & 1
    \end{tabular}
    \caption{Values of $T(n,k) = 6k(n-k) + 1$.
    See the sequence \href{https://oeis.org/A287326}{\texttt{A287326}} in OEIS
    ~\cite{oeis_numerical_triangle_row_sums_give_cubes}.}
    \label{tab:triangle_row_sums_give_cubes}
\end{table}

%Therefore, we have reached our base case by successfully interpolating the polynomial $n^3$.
%Fairly enough that the next curiosity would be
%\begin{question}
%    Well, if the relation~\eqref{eq:cube_identity} true for the polynomial $n^3$,
%    then is it true that~\eqref{eq:cube_identity} can be generalized for higher powers, e.g.\ for $n^4$ or $n^5$ similarly?
%    \label{question:higher_powers}
%\end{question}
%That was the next question, however without any expectation of the final form of generalized formula.
%Long story short, the answer to this question is also \textit{Yes}, by utilizing certain approaches
%in terms of systems of linear equations or recurrence formula, which is discussed.
%
%Let us begin from the background and history overview of systems of linear equations approach.
