Thus, the conjecture~\eqref{conj:odd-power-identity} is true

\begin{theorem}[Odd power identity]
    \label{theorem:odd-power-identity}
    There is a set of coefficients $\coeffA{m}{0}, \coeffA{m}{1}, \ldots, \coeffA{m}{m}$ such that
    \begin{align*}
        n^{2m+1} = \sum_{r=0}^{m} \sum_{k=1}^{n} \coeffA{m}{r} k^r (n-k)^r
    \end{align*}
\end{theorem}

\begin{definition}[Bivariate sum $T_m$]
    \label{def:bivariate-sum-Tm}
    \begin{align*}
        T_{m}(n,k) = \sum_{r=0}^{m} \coeffA{m}{r} k^r (n-k)^r
    \end{align*}
\end{definition}

\begin{proposition}[Symmetry of $T_m$]
    \label{prop:Tm-symmetry}
    For integers $n$ and $k$
    \begin{align*}
        T_{m} (n, k) = T_{m} (n, n-k)
    \end{align*}
\end{proposition}

\begin{proposition}[Recurrence for $T_m$]
    \label{prop:Tm-recurrence}
    For non-negative integers $m$ and $n$
    \begin{align*}
        T_{m} (n, k) = \sum_{t=1}^{m+1} (-1)^{t-1} \binom{m+1}{t} T_{m} (n-t, k)
    \end{align*}
\end{proposition}

\begin{proposition}[Odd power decomposition]
    \label{prop:odd-power-decomposition}
    For non-negative integers $m$ and $n$
    \begin{align*}
        n^{2m+1} = \sum_{k=1}^{n} \sum_{t=1}^{m+1} (-1)^{t-1} \binom{m+1}{t} T_{m} (n-t, k)
    \end{align*}
\end{proposition}

\begin{corollary}[Odd power decomposition $m-1$]
    \label{cor:odd-power-decomposition-m-1}
    \begin{align*}
        n^{2m-1} = \sum_{k=1}^{n} \sum_{t=1}^{m} (-1)^{t-1} \binom{m}{t} T_{m-1} (n-t, k)
    \end{align*}
\end{corollary}

\begin{proposition}[Odd power binomial form]
    \label{prop:odd-power-binomial}
    For integers $n$ and $a$ such that $n-2a \geq 0$
    \begin{align*}
    (n-2a)
        ^{2m+1} = \sum_{r=0}^{m} \sum_{k=a+1}^{n-a} \coeffA{m}{r} (k-a)^r (n-k-a)^r
    \end{align*}
\end{proposition}

\begin{corollary}[Odd power binomial form shifted]
    \label{prop:odd-power-binomial-shifted}
    For integers $n$ and $a$ such that $n-2a \geq 0$
    \begin{align*}
    (n-2a)
        ^{2m+1} = \sum_{r=0}^{m} \sum_{k=a}^{n-a-1} \coeffA{m}{r} (k-a)^r (n-k-a)^r
    \end{align*}
\end{corollary}

\begin{proposition}[Sum of odd powers]
    \label{prop:sum-of-odd-powers}
    \begin{align*}
        \sum_{t=1}^{n} t^{2m+1} = \sum_{t=1}^{n} \sum_{k=1}^{t} \sum_{r=0}^{m} \coeffA{m}{r} k^r (n-k)^r
    \end{align*}
\end{proposition}

