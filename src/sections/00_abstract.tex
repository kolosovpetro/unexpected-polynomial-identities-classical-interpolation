This manuscript originates from a classical interpolation problem: how to reconstruct the cubes $n^3$
from their finite differences.
The investigation leads to an unexpected identity expressing $n^3$ as sum of bivariate terms $k(n-k)$,
such that $n^3=\sum_{k=1}^{n} 6k(n-k) + 1$.
This identity serves as the base case for a more general identity for odd powers,
involving rational numbers $\mathbf{A}_{m,r}$ that is
$n^{2m+1} = \sum_{k=1}^{n} \sum_{r=0}^{m} \mathbf{A}_{m,r} k^r (n-k)^r$.
We evaluate the set of coefficients $\mathbf{A}_{m,0}, \mathbf{A}_{m,1}, \cdots, \mathbf{A}_{m,m}$
by solving a system of linear equations.
Furthermore, this work provides a recurrence relation
for coefficients $\mathbf{A}_{m,0}, \mathbf{A}_{m,1}, \cdots, \mathbf{A}_{m,m}$, by utilizing
generating functions.
The main results include odd power identities, identities for binomial forms,
and identities for sums of powers.
Apart that, we discuss the similarities between our findings and well-known results like Pascal's identity etc.
Afterward, the manuscript continues with related works that are based on our findings,
including approximation for powers,
derivatives, Faulhaber-like formulas.
This manuscript concludes with discussion of future research directions that include
the topics of integration into mathematical literature, approximation methods, combinatorial interpretations,
and $q$-derivatives.
