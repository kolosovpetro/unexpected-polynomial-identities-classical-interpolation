This manuscript originates from a classical interpolation problem: how to reconstruct the cubic polynomial $n^3$
from its finite differences.
The investigation leads to an unexpected identity expressing $n^3$ as a sum involving bivariate terms $k(n-k)$,
such that $n^3=\sum_{k=1}^{n} 6k(n-k) + 1$.
This identity for cubes serves as the base case of more general relation
including rational numbers $\mathbf{A}_{m,r}$, such that
$n^{2m+1} = \sum_{k=1}^{n} \sum_{r=0}^{m} \mathbf{A}_{m,r} k^r (n-k)^r$.
By solving certain system of linear equations,
we evaluate the set of coefficients $\mathbf{A}_{m,0}, \mathbf{A}_{m,1}, \cdots, \mathbf{A}_{m,m}$ such that
odd power identity $n^{2m+1} = \sum_{k=1}^{n} \sum_{r=0}^{m} \mathbf{A}_{m,r} k^r (n-k)^r$ holds.
Furthermore, we provide and discuss a recurrence relation
for coefficients $\mathbf{A}_{m,0}, \mathbf{A}_{m,1}, \cdots, \mathbf{A}_{m,m}$, by utilizing
specific generating function, enabling recursive construction of the coefficients.
The main results include odd power identities, identities for binomial forms,
and identities for sums of powers.
Apart that, we discuss the similar patterns between our finding and well-known results like Pascal's identity etc.
Afterward, we discuss related research that is based on our findings, this includes approximations for odd powers,
derivatives, Faulhaber-like formulas.
This manuscript concludes with discussion of future research directions that include
the topics of integration into mathematical literature, approximation methods, combinatorial interpretations,
and Q-Derivatives.
