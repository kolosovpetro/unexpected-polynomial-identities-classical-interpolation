Assume that the following odd power identity holds
\begin{align}
    \label{eq:odd-power-identity}
    n^{2m+1} = \sum_{r=0}^{m} \sum_{k=1}^{n} \coeffA{m}{r} k^r (n-k)^r
\end{align}
Our main goal is to identify the set of coefficients $\coeffA{m}{0}, \coeffA{m}{1}, \ldots, \coeffA{m}{m}$
such that identity above is true.

Although, the recurrence relation is already given at~\cite{alekseyev2018mathoverflow}, a few key points in proof
are worth to discuss additionally.

The main idea of Alekseyev's approach was to utilize a recurrence relation to evaluate the set of coefficients $\coeffA{m}{r}$
starting from the base case $\coeffA{m}{m}$ and then evaluating the next coefficient $\coeffA{m}{m-1}$
recursively, continuing similarly up to $\coeffA{m}{0}$.

We utilize Binomial theorem $(n-k)^r=\sum_{t=0}^{r} (-1)^t \binom{r}{t} n^{r-t} k^t$ and specific version
of Faulhaber's formula~\cite{beardon1996sums}
\begin{align*}
    \sum_{k=1}^{n} k^{p}
    = \frac{1}{p+1}\sum_{j=0}^{p} \binom{p+1}{j} \bernoulli{j} n^{p+1-j}
    &= \frac{1}{p+1} \left[ \sum_{j=0}^{p+1} \binom{p+1}{j} \bernoulli{j} n^{p+1-j} \right] - \frac{\bernoulli{p+1}}{p+1} \\
    &= \frac{1}{p+1} \left[ \sum_{j} \binom{p+1}{j} \bernoulli{j} n^{p+1-j} \right] - \frac{\bernoulli{p+1}}{p+1}
\end{align*}
The reason we use modified version of Faulhaber's formula is because we want to omit summation bounds, for simplicity.
This would help us to collapse the common terms across complex sums, see also~\cite[~p. 2]{knuth1992two}.
Therefore, we expand the sum $\sum_{k=1}^{n} k^{r} (n-k)^{r}$ using Binomial theorem
\begin{align*}
    \sum_{k=1}^{n} k^{r} (n-k)^{r} = \sum_{t=0}^{r} (-1)^t \binom{r}{t} n^{r-t} \sum_{k=1}^{n} k^{t+r}
\end{align*}
By applying Faulhaber's formula above, we obtain
\begin{align*}
    \sum_{k=1}^{n} k^{r} (n-k)^{r}
    = \sum_{t=0}^{r} (-1)^t \binom{r}{t} n^{r-t} \left[ \left( \frac{1}{t+r+1} \sum_{j} \binom{t+r+1}{j} \bernoulli{j} n^{t+r+1-j} \right) - \frac{\bernoulli{t+r+1}}{t+r+1} \right]
\end{align*}
By moving the common term $\frac{(-1)^t}{t+r+1}$ out of brackets
\begin{align*}
    \sum_{k=1}^{n} k^{r} (n-k)^{r}
    = \sum_{t=0}^{r} \binom{r}{t} \frac{(-1)^t}{t+r+1} \left[ \sum_{j} \binom{t+r+1}{j} \bernoulli{j} n^{2r+1-j} - \bernoulli{t+r+1} n^{r-t} \right]
\end{align*}
By expanding the brackets
\begin{align*}
    \sum_{k=1}^{n} k^{r} (n-k)^{r}
    &= \left[ \sum_{t=0}^{r} \binom{r}{t} \frac{(-1)^t}{t+r+1} \sum_{j} \binom{t+r+1}{j} \bernoulli{j} n^{2r+1-j}  \right] \\
    &- \left[ \sum_{t=0}^{r} \binom{r}{t} \frac{(-1)^t}{t+r+1} \bernoulli{t+r+1} n^{r-t} \right]
\end{align*}
By moving the sum in $j$ and omitting summation bounds in $t$
\begin{align*}
    \sum_{k=1}^{n} k^{r} (n-k)^{r}
    = \left[ \sum_{j, t} \binom{r}{t} \frac{(-1)^t}{t+r+1} \binom{t+r+1}{j} \bernoulli{j} n^{2r+1-j}  \right]
    - \left[ \sum_{t} \binom{r}{t} \frac{(-1)^t}{t+r+1} \bernoulli{t+r+1} n^{r-t} \right]
\end{align*}
By rearranging the sums we obtain
\begin{align}
    \label{eq:rearranging-terms}
    \sum_{k=1}^{n} k^{r} (n-k)^{r}
    &= \left[ \sum_{j} \bernoulli{j} n^{2r+1-j} \sum_{t} \binom{r}{t} \frac{(-1)^t}{t+r+1} \binom{t+r+1}{j}  \right] \\
    &- \left[ \sum_{t} \binom{r}{t} \frac{(-1)^t}{t+r+1} \bernoulli{t+r+1} n^{r-t} \right] \nonumber
\end{align}
We can notice that
\begin{equation}
    \label{eq:combinatorial-identity}
    \sum_{t} \binom{r}{t} \frac{(-1)^t}{r+t+1} \binom{r+t+1}{j}
    =\begin{cases}
         \frac{1}{(2r+1) \binom{2r}r} & \text{if } j=0\\
         \frac{(-1)^r}{j} \binom{r}{2r-j+1} & \text{if } j>0
    \end{cases}
\end{equation}
An elegant proof of the binomial identity~\eqref{eq:combinatorial-identity} is done by Markus Scheuer in~\cite{scheuer2023mathstackexchange}.
In particular, the equation~\eqref{eq:combinatorial-identity} is zero for $0< t \leq j$.
To utilize the equation~\eqref{eq:combinatorial-identity}, we have to move $j=0$ out of summation
in~\eqref{eq:rearranging-terms} to avoid division by zero in $\frac{(-1)^r}{j}$.
Therefore,
\begin{equation*}
    \begin{split}
        \sum_{k=1}^{n} k^{r} (n-k)^{r}
        &= \frac{1}{(2r+1) \binom{2r}r} n^{2r+1}
        + \left[ \sum_{j = 1}^{\infty} \bernoulli{j} n^{2r+1-j} \sum_{t} \binom{r}{t} \frac{(-1)^t}{t+r+1} \binom{t+r+1}{j} \right] \\
        &- \left[ \sum_{t} \binom{r}{t} \frac{(-1)^t}{t+r+1} \bernoulli{t+r+1} n^{r-t} \right]
    \end{split}
\end{equation*}
Hence, we simplify the equation above by using~\eqref{eq:combinatorial-identity} so that
\begin{equation*}
    \begin{split}
        \sum_{k=1}^{n} k^{r} (n-k)^{r}
        &= \frac{1}{(2r+1) \binom{2r}r} n^{2r+1}
        + \underbrace{\left[ \sum_{j = 1}^{\infty} \frac{(-1)^r}{j} \binom{r}{2r-j+1} \bernoulli{j} n^{2r-j+1} \right]}_{(\star)} \\
        &- \underbrace{\left[ \sum_{t} \binom{r}{t} \frac{(-1)^t}{t+r+1} \bernoulli{t+r+1} n^{r-t} \right]}_{(\diamond)}
    \end{split}
\end{equation*}
By introducing $\ell=2r-j+1$ to $(\star)$ and $\ell=r-t$ to $(\diamond)$
we collapse the common terms across two sums
\begin{equation*}
    \begin{split}
        \sum_{k=1}^{n} k^{r} (n-k)^{r}
        &= \frac{1}{(2r+1) \binom{2r}r} n^{2r+1}
        + \left[ \sum_{\ell} \frac{(-1)^r}{2r+1-\ell} \binom{r}{\ell} \bernoulli{2r+1-\ell} n^{\ell} \right] \\
        &- \left[ \sum_{\ell} \binom{r}{\ell} \frac{(-1)^{r-\ell}}{2r+1-\ell} \bernoulli{2r+1-\ell} n^{\ell} \right]\\
        &= \frac{1}{(2r+1) \binom{2r}r} n^{2r+1} + 2 \sum_{\mathrm{odd \; \ell}} \frac{(-1)^r}{2r+1-\ell} \binom{r}{\ell} \bernoulli{2r+1-\ell} n^{\ell}
    \end{split}
\end{equation*}
Assuming that $\coeffA{m}{r}$ is defined by the odd-power identity~\eqref{eq:odd-power-identity},
we obtain the following relation for polynomials in $n$
\begin{equation*}
    \sum_{r=0}^{m} \coeffA{m}{r} \frac{1}{(2r+1) \binom{2r}r} n^{2r+1}
    + 2 \sum_{r=0}^{m} \sum_{\mathrm{odd \; \ell}} \coeffA{m}{r} \frac{(-1)^r}{2r+1-\ell} \binom{r}{\ell} \bernoulli{2r+1-\ell} n^{\ell}
    \equiv n^{2m+1}
\end{equation*}
Replacing odd $\ell$ by $\ell = 2k+1$ we get
\begin{align*}
    \sum_{r=0}^{m} \coeffA{m}{r} \frac{1}{(2r+1) \binom{2r}{r}} n^{2r+1} + 2 \sum_{r=0}^{m} \sum_{k=0}^{\infty} \coeffA{m}{r} \frac{(-1)^r}{2r-2k} \binom{r}{2k+1} \bernoulli{2r-2k} n^{2k+1}  \equiv n^{2m+1}
\end{align*}
By simplifying the term $2$
\begin{align}
    \label{eq:main_relation}
    \sum_{r=0}^{m} \coeffA{m}{r} \frac{1}{(2r+1) \binom{2r}{r}} n^{2r+1} + \sum_{r=0}^{m} \sum_{k=0}^{\infty} \coeffA{m}{r} \frac{(-1)^r}{r-k} \binom{r}{2k+1} \bernoulli{2r-2k} n^{2k+1}  \equiv n^{2m+1}
\end{align}
Basically, the relation~\eqref{eq:main_relation} is the generating function we utilize to
evaluate the values of $\coeffA{m}{0}, \coeffA{m}{1}, \ldots, \coeffA{m}{m}$.
We now fix the unused values of $\coeffA{m}{r}$ so that $\coeffA{m}{r} = 0$ for every $r < 0$ or $r > m$.

Taking the coefficient of $n^{2m+1}$ in~\eqref{eq:main_relation} yields
\begin{align*}
    \coeffA{m}{m} = (2m+1)\binom{2m}{m}
\end{align*}
because $\coeffA{m}{m} \frac{1}{(2m+1) \binom{2m}{m}} = 1$.

That's may not be immediately clear why the coefficient of $n^{2m+1}$ is $(2m+1)\binom{2m}{m}$.
To extract the coefficient of $n^{2m+1}$ from the expression~\eqref{eq:main_relation},
we isolate the relevant terms by setting $r = m$ in the first sum,
and $k = m$ in the second sum.
This gives
\begin{align*}
[n^{2m+1}] &\left(
                \sum_{r=0}^{m} \coeffA{m}{r} \frac{1}{(2r+1) \binom{2r}{r}} n^{2r+1}
                + \sum_{r=0}^{m} \sum_{k=0}^{\infty} \coeffA{m}{r} \frac{(-1)^r}{r-k} \binom{r}{2k+1} \bernoulli{2r - 2k} n^{2k+1}
                - n^{2m+1}
\right) \\
&= \coeffA{m}{m} \frac{1}{(2m+1) \binom{2m}{m}}
+ \sum_{r=0}^{m} \coeffA{m}{r} \frac{(-1)^r}{r - m} \binom{r}{2m+1} \bernoulli{2r - 2m}
- 1
\end{align*}
We observe that the sum
\begin{align*}
    \sum_{r=0}^{m} \coeffA{m}{r} \frac{(-1)^r}{r - m} \binom{r}{2m+1} \bernoulli{2r - 2m}
\end{align*}
does not contribute to the determination of the coefficients because the binomial coefficient
$\binom{r}{2m+1}$ vanishes for all $r \leq m$.
Consequently, all terms in the sum are zero.
Thus,
\begin{align*}
    \coeffA{m}{m} \frac{1}{(2m+1) \binom{2m}{m}}  - 1 = 0 \implies \coeffA{m}{m} = (2m+1) \binom{2m}{m}
\end{align*}

Taking the coefficient of $n^{2d+1}$ for an integer $d$ in the range $\frac{m}{2} \leq d \leq m-1$ in~\eqref{eq:main_relation} gives
\begin{align*}
[n^{2d+1}] &\left( \sum_{r=0}^{m} \coeffA{m}{r} \frac{1}{(2r+1) \binom{2r}{r}} n^{2r+1} + \sum_{r=0}^{m} \sum_{k=0}^{\infty} \coeffA{m}{r} \frac{(-1)^r}{r-k} \binom{r}{2k+1} \bernoulli{2r-2k} n^{2k+1} - n^{2m+1} \right) \\
&= \coeffA{m}{d} \frac{1}{(2d+1) \binom{2d}{d}} + \sum_{r=0}^m \coeffA{m}{r} \frac{(-1)^r}{r-d} \binom{r}{2d+1} \bernoulli{2r-2d}.
\end{align*}
For every $\frac{m}{2} \leq d$, the binomial coefficient $\binom{r}{2d+1}$ vanishes, because for all $r \leq m$
holds $r < 2d+1$.
As a particular example, when $r = m$ and $d = \frac{m}{2}$, we have
\begin{align*}
    \binom{m}{m+1} = 0.
\end{align*}
Therefore, the entire sum involving $\binom{r}{2d+1}$ vanishes, and we conclude
\begin{align*}
    \coeffA{m}{d} \frac{1}{(2d+1) \binom{2d}{d}} = 0 \implies \coeffA{m}{d} = 0.
\end{align*}
Hence, for all integers $d$ such that $\frac{m}{2} \leq d \leq m-1$, the coefficient $\coeffA{m}{d} = 0$.
In contrast, for values $d \leq \frac{m}{2} - 1$, the binomial coefficient $\binom{r}{2d+1}$ can be nonzero; for instance, if $r = m$ and $d = \frac{m}{2} - 1$, then
\begin{align*}
    \binom{m}{m - 1} \neq 0,
\end{align*}
allowing the corresponding terms to contribute to the determination of $\coeffA{m}{d}$.

Taking the coefficient of $n^{2d+1}$ for $d$ in the range $\frac{m}{4} \leq d < \frac{m}{2}$ in~\eqref{eq:main_relation}, we obtain
\begin{align*}
    \coeffA{m}{d} \frac{1}{(2d+1) \binom{2d}{d}}
    + 2 (2m+1) \binom{2m}{m} \binom{m}{2d+1} \frac{(-1)^m}{2m - 2d} \bernoulli{2m - 2d} = 0.
\end{align*}
Solving for $\coeffA{m}{d}$ yields
\begin{equation*}
    \coeffA{m}{d}
    = (-1)^{m-1} \frac{(2m+1)!}{d! \, d! \, m! \, (m - 2d - 1)!} \cdot \frac{1}{m - d} \bernoulli{2m - 2d}.
\end{equation*}

Proceeding recursively, we can compute each coefficient $\coeffA{m}{r}$ for integers $r$ in the ranges
\begin{align*}
    \frac{m}{2^{s+1}} \leq r < \frac{m}{2^s}, \quad \text{for } s = 1, 2, \ldots
\end{align*}
by using previously computed values $\coeffA{m}{d}$ for $d > r$, via the relation
\begin{equation*}
    \coeffA{m}{r} =
    (2r+1) \binom{2r}{r} \sum_{d = 2r+1}^{m}
    \coeffA{m}{d} \binom{d}{2r+1} \frac{(-1)^{d-1}}{d - r} \bernoulli{2d - 2r}.
\end{equation*}

Finally, we define the following recurrence relation for coefficients $\coeffA{m}{r}$
\begin{definition} (Definition of coefficient $\coeffA{m}{r}$.)
    \begin{equation}
        \label{eq:definition_coefficient_a}
        \coeffA{m}{r} =
        \begin{cases}
        (2r+1)
            \binom{2r}{r} & \mathrm{if} \; r=m \\
            (2r+1) \binom{2r}{r} \sum_{d = 2r+1}^{m} \coeffA{m}{d} \binom{d}{2r+1} \frac{(-1)^{d-1}}{d-r}
            \bernoulli{2d-2r} & \mathrm{if} \; 0 \leq r<m \\
            0 & \mathrm{if} \; r<0 \; \mathrm{or} \; r>m
        \end{cases}
    \end{equation}
\end{definition}
where $\bernoulli{t}$ are Bernoulli numbers~\cite{bateman1953higher}.
It is assumed that $\bernoulli{1}=\frac{1}{2}$.
For example,
\begin{table}[H]
    \begin{center}
        \setlength\extrarowheight{-6pt}
        \begin{tabular}{c|cccccccc}
            $m/r$ & 0 & 1       & 2      & 3      & 4   & 5    & 6     & 7 \\ [3px]
            \hline
            0     & 1 &         &        &        &     &      &       &       \\
            1     & 1 & 6       &        &        &     &      &       &       \\
            2     & 1 & 0       & 30     &        &     &      &       &       \\
            3     & 1 & -14     & 0      & 140    &     &      &       &       \\
            4     & 1 & -120    & 0      & 0      & 630 &      &       &       \\
            5     & 1 & -1386   & 660    & 0      & 0   & 2772 &       &       \\
            6     & 1 & -21840  & 18018  & 0      & 0   & 0    & 12012 &       \\
            7     & 1 & -450054 & 491400 & -60060 & 0   & 0    & 0     & 51480
        \end{tabular}
    \end{center}
    \caption{Coefficients $\coeffA{m}{r}$. See OEIS sequences
    ~\cite{oeis_numerators_of_the_coefficient_a_m_r,oeis_denominators_of_the_coefficient_a_m_r}.}
    \label{tab:table_of_coefficients_a}
\end{table}

Properties of the coefficients $\coeffA{m}{r}$
\begin{itemize}
    \item $\coeffA{m}{m} = \binom{2m}{m}$
    \item $\coeffA{m}{r} = 0$ for $m < 0$ and $r > m$
    \item $\coeffA{m}{r} = 0$ for $r < 0$
    \item $\coeffA{m}{r} = 0$ for $\frac{m}{2} \leq r < m$
    \item $\coeffA{m}{0} = 1$ for $m \geq 0$
    \item $\coeffA{m}{r}$ are integers for $m \leq 11$
    \item Row sums: $\sum_{r=0}^{m} \coeffA{m}{r} = 2^{2m+1} - 1$
\end{itemize}
