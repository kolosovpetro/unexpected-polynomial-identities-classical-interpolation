In 2018, the recurrence relation~\cite{alekseyev2018mathoverflow} that evaluates the coefficients $\coeffA{m}{r}$ for
non-negative integer $m$ was provided by Max Alekseyev, George Washington University.
The main idea of Alekseyev's approach was to utilize a generating function to evaluate the set of coefficients $\coeffA{m}{r}$
starting from the base case $\coeffA{m}{m}$ and then evaluating the next coefficient $\coeffA{m}{m-1}$
recursively, and so on up to $\coeffA{m}{0}$.
We utilize Binomial theorem $(n-k)^r=\sum_{t=0}^{r} (-1)^t \binom{r}{t} n^{r-t} k^t$ and a specific version
of Faulhaber's formula~\cite{beardon1996sums} with upper bound $p+1$
\begin{align*}
    \sum_{k=1}^{n} k^{p}
    = \frac{1}{p+1}\sum_{j=0}^{p} \binom{p+1}{j} \bernoulli{j} n^{p+1-j}
    &= \frac{1}{p+1} \left[ \sum_{j=0}^{p+1} \binom{p+1}{j} \bernoulli{j} n^{p+1-j} \right] - \frac{\bernoulli{p+1}}{p+1}
%    &= \frac{1}{p+1} \left[ \sum_{j} \binom{p+1}{j} \bernoulli{j} n^{p+1-j} \right] - \frac{\bernoulli{p+1}}{p+1}
\end{align*}
The reason we use the Faulhaber's formula above is because we tend to omit summation bounds, for simplicity.
This helps us to collapse the common terms across complex sums, because now can extend the sum over all integers $j$,
while only finitely many terms $\binom{p+1}{j}$ are non-zero, see also~\cite[~p. 2]{knuth1992two}.
Hence,
\begin{align}
    \label{eq:modified-faulhabers-formula}
    \sum_{k=1}^{n} k^{p}
    = \frac{1}{p+1} \left[ \sum_{j} \binom{p+1}{j} \bernoulli{j} n^{p+1-j} \right] - \frac{\bernoulli{p+1}}{p+1}
\end{align}
Now we expand the sum $\sum_{k=1}^{n} k^{r} (n-k)^{r}$ using Binomial theorem
\begin{align*}
    \sum_{k=1}^{n} k^{r} (n-k)^{r} = \sum_{t=0}^{r} (-1)^t \binom{r}{t} n^{r-t} \sum_{k=1}^{n} k^{t+r}
\end{align*}
By applying Faulhaber's formula~\eqref{eq:modified-faulhabers-formula}, we obtain
\begin{align*}
    \sum_{k=1}^{n} k^{r} (n-k)^{r}
    = \sum_{t=0}^{r} (-1)^t \binom{r}{t} n^{r-t} \left[ \left( \frac{1}{t+r+1} \sum_{j} \binom{t+r+1}{j} \bernoulli{j} n^{t+r+1-j} \right) - \frac{\bernoulli{t+r+1}}{t+r+1} \right]
\end{align*}
By moving the common term $\frac{(-1)^t}{t+r+1}$ out of brackets
\begin{align*}
    \sum_{k=1}^{n} k^{r} (n-k)^{r}
    = \sum_{t=0}^{r} \binom{r}{t} \frac{(-1)^t}{t+r+1} \left[ \sum_{j} \binom{t+r+1}{j} \bernoulli{j} n^{2r+1-j} - \bernoulli{t+r+1} n^{r-t} \right]
\end{align*}
By expanding the brackets
\begin{align*}
    \sum_{k=1}^{n} k^{r} (n-k)^{r}
    &= \left[ \sum_{t=0}^{r} \binom{r}{t} \frac{(-1)^t}{t+r+1} \sum_{j} \binom{t+r+1}{j} \bernoulli{j} n^{2r+1-j}  \right] \\
    &- \left[ \sum_{t=0}^{r} \binom{r}{t} \frac{(-1)^t}{t+r+1} \bernoulli{t+r+1} n^{r-t} \right]
\end{align*}
By moving the sum in $j$ and omitting summation bounds in $t$
\begin{align*}
    \sum_{k=1}^{n} k^{r} (n-k)^{r}
    = \left[ \sum_{j, t} \binom{r}{t} \frac{(-1)^t}{t+r+1} \binom{t+r+1}{j} \bernoulli{j} n^{2r+1-j}  \right]
    - \left[ \sum_{t} \binom{r}{t} \frac{(-1)^t}{t+r+1} \bernoulli{t+r+1} n^{r-t} \right]
\end{align*}
By rearranging the sums we obtain
\begin{align}
    \label{eq:rearranging-terms}
    \sum_{k=1}^{n} k^{r} (n-k)^{r}
    &= \left[ \sum_{j} \bernoulli{j} n^{2r+1-j} \sum_{t} \binom{r}{t} \frac{(-1)^t}{t+r+1} \binom{t+r+1}{j}  \right] \\
    &- \left[ \sum_{t} \binom{r}{t} \frac{(-1)^t}{t+r+1} \bernoulli{t+r+1} n^{r-t} \right] \nonumber
\end{align}
We can notice that
\begin{lemma}
    \label{lem:combinatorial-identity}
    For integers $r, j$
    \begin{align*}
        \sum_{t} \binom{r}{t} \frac{(-1)^t}{r+t+1} \binom{r+t+1}{j}
        =\begin{cases}
             \frac{1}{(2r+1) \binom{2r}r} & \text{if } j=0\\
             \frac{(-1)^r}{j} \binom{r}{2r-j+1} & \text{if } j>0
        \end{cases}
    \end{align*}
    \begin{proof}
        An elegant proof is done by Markus Scheuer in~\cite{scheuer2023mathstackexchange}.
    \end{proof}
\end{lemma}
In particular, the sum above is zero for $0< j \leq r$.
To simplify~\eqref{eq:rearranging-terms} using lemma~\eqref{lem:combinatorial-identity}, we have to move $j=0$ out of
the sum $\Sigma$
in~\eqref{eq:rearranging-terms} to avoid division by zero in $\frac{(-1)^r}{j}$.
Therefore,
\begin{equation*}
    \begin{split}
        \sum_{k=1}^{n} k^{r} (n-k)^{r}
        &= \frac{1}{(2r+1) \binom{2r}r} n^{2r+1}
        + \left[ \sum_{j = 1}^{\infty} \bernoulli{j} n^{2r+1-j} \sum_{t} \binom{r}{t} \frac{(-1)^t}{t+r+1} \binom{t+r+1}{j} \right] \\
        &- \left[ \sum_{t} \binom{r}{t} \frac{(-1)^t}{t+r+1} \bernoulli{t+r+1} n^{r-t} \right]
    \end{split}
\end{equation*}
Hence, we simplify the equation~\eqref{eq:rearranging-terms} by using lemma~\eqref{lem:combinatorial-identity}
\begin{equation*}
    \begin{split}
        \sum_{k=1}^{n} k^{r} (n-k)^{r}
        &= \frac{1}{(2r+1) \binom{2r}r} n^{2r+1}
        + \left[ \sum_{j=1}^{\infty} \frac{(-1)^r}{j} \binom{r}{2r-j+1} \bernoulli{j} n^{2r-j+1} \right] \\
        &- \left[ \sum_{t} \binom{r}{t} \frac{(-1)^t}{t+r+1} \bernoulli{t+r+1} n^{r-t} \right]
    \end{split}
\end{equation*}
By setting $\ell=2r-j+1$ to the sum $\sum_{j=1}^{\infty}$, and $\ell=r-t$ to the sum $\sum_{t}$,
we collapse the common terms across two sums, thus
\begin{align*}
    \sum_{k=1}^{n} k^{r} (n-k)^{r}
    &= \frac{1}{(2r+1) \binom{2r}r} n^{2r+1}
    + \left[ \sum_{\ell} \frac{(-1)^r}{2r+1-\ell} \binom{r}{\ell} \bernoulli{2r+1-\ell} n^{\ell} \right] \\
    &- \left[ \sum_{\ell} \binom{r}{\ell} \frac{(-1)^{r-\ell}}{2r+1-\ell} \bernoulli{2r+1-\ell} n^{\ell} \right]\\
    &= \frac{1}{(2r+1) \binom{2r}r} n^{2r+1} + 2 \sum_{\mathrm{odd \; \ell}} \frac{(-1)^r}{2r+1-\ell} \binom{r}{\ell} \bernoulli{2r+1-\ell} n^{\ell}
\end{align*}
By replacing odd $\ell$ by $\ell = 2k+1$, and by simplifying 2's, we get
\begin{proposition}[Bivariate Faulhaber's Formula]
    \label{prop:bivariate-faulhabers-formula}
    \begin{align*}
        \sum_{k=1}^{n} k^{r} (n-k)^{r}
        = \frac{1}{(2r+1) \binom{2r}r} n^{2r+1}
        + \sum_{k=0}^{\infty} \frac{(-1)^r}{r-k} \binom{r}{2k+1} \bernoulli{2r-2k} n^{2k+1}
    \end{align*}
\end{proposition}

Assuming that $\coeffA{m}{r}$ is defined
by the odd-power identity in conjecture~\eqref{conj:odd-power-identity},
we obtain the following relation for polynomials in $n$
\begin{align}
    \label{eq:main_relation}
    \sum_{r=0}^{m} \coeffA{m}{r} \frac{1}{(2r+1) \binom{2r}{r}} n^{2r+1} + \sum_{r=0}^{m} \sum_{k=0}^{\infty} \coeffA{m}{r} \frac{(-1)^r}{r-k} \binom{r}{2k+1} \bernoulli{2r-2k} n^{2k+1}  \equiv n^{2m+1}
\end{align}
Basically, the relation~\eqref{eq:main_relation} is the generating function we utilize to
evaluate the values of $\coeffA{m}{m}, \coeffA{m}{m-1}, \ldots, \coeffA{m}{0}$.
We now fix the unused values of $\coeffA{m}{r}$ so that $\coeffA{m}{r} = 0$ for every $r < 0$ or $r > m$.

Taking the coefficient of $n^{2m+1}$ in~\eqref{eq:main_relation} yields
\begin{align*}
    \coeffA{m}{m} = (2m+1)\binom{2m}{m}
\end{align*}
because $\coeffA{m}{m} \frac{1}{(2m+1) \binom{2m}{m}} = 1$.

That's may not be immediately clear why the coefficient of $n^{2m+1}$ is $(2m+1)\binom{2m}{m}$.
To extract the coefficient of $n^{2m+1}$ from the expression~\eqref{eq:main_relation},
we isolate the relevant terms by setting $r = m$ in the first sum,
and $k = m$ in the second sum.
This gives
\begin{align*}
[n^{2m+1}]
    &\left(
         \sum_{r=0}^{m} \coeffA{m}{r} \frac{1}{(2r+1) \binom{2r}{r}} n^{2r+1}
         + \sum_{r=0}^{m} \sum_{k=0}^{\infty} \coeffA{m}{r} \frac{(-1)^r}{r-k} \binom{r}{2k+1} \bernoulli{2r - 2k} n^{2k+1}
         - n^{2m+1}
    \right) \\
    &= \coeffA{m}{m} \frac{1}{(2m+1) \binom{2m}{m}}
    + \sum_{r=0}^{m} \coeffA{m}{r} \frac{(-1)^r}{r - m} \binom{r}{2m+1} \bernoulli{2r - 2m}
    - 1
\end{align*}
We observe that the sum
\begin{align*}
    \sum_{r=0}^{m} \coeffA{m}{r} \frac{(-1)^r}{r - m} \binom{r}{2m+1} \bernoulli{2r - 2m}
\end{align*}
does not contribute to the determination of the coefficients $\coeffA{m}{r}$, because the binomial coefficient
$\binom{r}{2m+1}$ vanishes for all $r \leq m$.
Consequently, all terms in the sum are zero.
Thus,
\begin{align*}
    \coeffA{m}{m} \frac{1}{(2m+1) \binom{2m}{m}}  - 1 = 0 \implies \coeffA{m}{m} = (2m+1) \binom{2m}{m}
\end{align*}

Taking the coefficient of $n^{2d+1}$ for an integer $d$ in the range $\frac{m}{2} \leq d \leq m-1$ in~\eqref{eq:main_relation} gives
\begin{align*}
[n^{2d+1}]
    &\left( \sum_{r=0}^{m} \coeffA{m}{r} \frac{1}{(2r+1) \binom{2r}{r}} n^{2r+1} + \sum_{r=0}^{m} \sum_{k=0}^{\infty} \coeffA{m}{r} \frac{(-1)^r}{r-k} \binom{r}{2k+1} \bernoulli{2r-2k} n^{2k+1} - n^{2m+1} \right) \\
    &= \coeffA{m}{d} \frac{1}{(2d+1) \binom{2d}{d}} + \sum_{r=0}^m \coeffA{m}{r} \frac{(-1)^r}{r-d} \binom{r}{2d+1} \bernoulli{2r-2d}.
\end{align*}
For every $\frac{m}{2} \leq d$, the binomial coefficient $\binom{r}{2d+1}$ vanishes, because for all $r \leq m$
holds $r < 2d+1$.
As a particular example, when $r = m$ and $d = \frac{m}{2}$, we have
\begin{align*}
    \binom{m}{m+1} = 0.
\end{align*}
Therefore, the entire sum involving $\binom{r}{2d+1}$ vanishes, and we conclude
\begin{align*}
    \coeffA{m}{d} \frac{1}{(2d+1) \binom{2d}{d}} = 0 \implies \coeffA{m}{d} = 0.
\end{align*}
Hence, for all integers $d$ such that $\frac{m}{2} \leq d \leq m-1$, the coefficient $\coeffA{m}{d} = 0$.
In contrast, for values $d \leq \frac{m}{2} - 1$, the binomial coefficient $\binom{r}{2d+1}$ can be nonzero; for instance, if $r = m$ and $d = \frac{m}{2} - 1$, then
\begin{align*}
    \binom{m}{m - 1} \neq 0,
\end{align*}
allowing the corresponding terms to contribute to the determination of $\coeffA{m}{d}$.

Taking the coefficient of $n^{2d+1}$ for $d$ in the range $\frac{m}{4} \leq d < \frac{m}{2}$ in~\eqref{eq:main_relation}, we obtain
\begin{align*}
    \coeffA{m}{d} \frac{1}{(2d+1) \binom{2d}{d}}
    + 2 (2m+1) \binom{2m}{m} \binom{m}{2d+1} \frac{(-1)^m}{2m - 2d} \bernoulli{2m - 2d} = 0.
\end{align*}
Solving for $\coeffA{m}{d}$ yields
\begin{equation*}
    \coeffA{m}{d}
    = (-1)^{m-1} \frac{(2m+1)!}{d! \, d! \, m! \, (m - 2d - 1)!} \cdot \frac{1}{m - d} \bernoulli{2m - 2d}.
\end{equation*}

Proceeding recursively, we can compute each coefficient $\coeffA{m}{r}$ for integers $r$ in the ranges
$\frac{m}{2^{s+1}} \leq r < \frac{m}{2^s}, \quad \text{for } s = 1, 2, \ldots$,
by using previously computed values $\coeffA{m}{d}$ for $d > r$, via the relation
\begin{equation*}
    \coeffA{m}{r} =
    (2r+1) \binom{2r}{r} \sum_{d = 2r+1}^{m}
    \coeffA{m}{d} \binom{d}{2r+1} \frac{(-1)^{d-1}}{d - r} \bernoulli{2d - 2r}.
\end{equation*}

Finally, we define the following recurrence relation for coefficients $\coeffA{m}{r}$
\begin{proposition}
    For integers $m$ and $r$
    \label{prop:coefficients_a}
    \begin{align*}
        \coeffA{m}{r} =
        \begin{cases}
        (2r+1)
            \binom{2r}{r} & \mathrm{if} \; r=m \\
            (2r+1) \binom{2r}{r} \sum_{d = 2r+1}^{m} \coeffA{m}{d} \binom{d}{2r+1} \frac{(-1)^{d-1}}{d-r}
            \bernoulli{2d-2r} & \mathrm{if} \; 0 \leq r<m \\
            0 & \mathrm{if} \; r<0 \; \mathrm{or} \; r>m
        \end{cases}
    \end{align*}
    where $\bernoulli{t}$ are Bernoulli numbers~\cite{bateman1953higher}.
    It is assumed that $\bernoulli{1}=\frac{1}{2}$.
\end{proposition}
For example,
\begin{table}[H]
    \begin{center}
        \setlength\extrarowheight{-6pt}
        \begin{tabular}{c|cccccccc}
            $m/r$ & 0 & 1       & 2      & 3      & 4   & 5    & 6     & 7 \\ [3px]
            \hline
            0     & 1 &         &        &        &     &      &       &       \\
            1     & 1 & 6       &        &        &     &      &       &       \\
            2     & 1 & 0       & 30     &        &     &      &       &       \\
            3     & 1 & -14     & 0      & 140    &     &      &       &       \\
            4     & 1 & -120    & 0      & 0      & 630 &      &       &       \\
            5     & 1 & -1386   & 660    & 0      & 0   & 2772 &       &       \\
            6     & 1 & -21840  & 18018  & 0      & 0   & 0    & 12012 &       \\
            7     & 1 & -450054 & 491400 & -60060 & 0   & 0    & 0     & 51480
        \end{tabular}
    \end{center}
    \caption{Coefficients $\coeffA{m}{r}$. See OEIS sequences
    ~\cite{oeis_numerators_of_the_coefficient_a_m_r,oeis_denominators_of_the_coefficient_a_m_r}.}
    \label{tab:table_of_coefficients_a}
\end{table}

Properties of the coefficients $\coeffA{m}{r}$
\begin{itemize}
    \item $\coeffA{m}{m} = \binom{2m}{m}$.
    \item $\coeffA{m}{r} = 0$ for $r < 0$ and $r > m$.
    \item $\coeffA{m}{r} = 0$ for $m < 0$.
    \item $\coeffA{m}{r} = 0$ for $\frac{m}{2} \leq r < m$.
    \item $\coeffA{m}{0} = 1$ for $m \geq 0$.
    \item $\coeffA{m}{r}$ are all integers up to row $m = 11$.
    \item Row sums: $\sum_{r=0}^{m} \coeffA{m}{r} = 2^{2m+1} - 1$.
\end{itemize}
For instance,
\begin{align*}
    n^3 &= \sum_{k=1}^{n} 6k(n-k) + 1 \\
    n^5 &= \sum_{k=1}^{n} 30k^2(n-k)^2 + 1 \\
    n^7 &= \sum_{k=1}^{n} 140 k^3 (n-k)^3 - 14k(n-k) + 1 \\
    n^9 &= \sum_{k=1}^{n} 630 k^4(n-k)^4 - 120k(n-k) + 1 \\
    n^{11} &= \sum_{k=1}^{n} 2772 k^5(n-k)^5 + 660 k^2(n-k)^2 - 1386k(n-k) + 1 \\
    n^{13} &= \sum_{k=1}^{n} 51480 k^7(n-k)^7 - 60060 k^3(n-k)^3 + 491400k^2(n-k)^{2} - 450054k(n-k) + 1
\end{align*}
