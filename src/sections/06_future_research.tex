Several promising directions emerge from the findings of this manuscript:

\subsection{Integration into mathematical literature}
\label{subsec:integration-into-mathematical-literature}
The identities presented in this work do not appear in standard mathematical references,
despite their elementary nature and apparent classical flavor.
Notably, related sequences are absent from major repositories such as the OEIS\@.
Future work should investigate the originality of these results and aim to contextualize
them within the broader mathematical framework.

\subsection{Extension of approximation methods}
\label{subsec:extension-of-approximation-methods}
The approximation technique developed in~\cite{kolosov2025efficient} is generalizable
to a broader class of polynomials.
In particular, by leveraging the symmetry property~\eqref{prop:Tm-symmetry},
one could explore alternative summation domains for the polynomials $P(m, X, N)$.

\subsection{Combinatorial interpretation of $T_m(n,k)$}
\label{subsec:combinatorial-interpretation-of-T}
The polynomial family $T_m(n,k)$, introduced in~\eqref{def:bivariate-sum-Tm},
currently lacks a clear combinatorial interpretation.
Understanding its structural or enumerative significance would deepen insight into
the algebraic identities presented.

\subsection{Connection with finite differences and derivatives}
\label{subsec:connection-with-finite-differences-and-derivatives}
The binomial form of the odd power identity~\eqref{prop:binomial-form} offers a mechanism
to express both finite differences and classical derivatives of odd powers in terms
of the coefficients $\coeffA{m}{r}$.

\subsection{$q$-derivative representation}
\label{subsec:q-derivative-representation}
The general identity~\eqref{theorem:odd-power-identity} also suggests a natural
expression for $q$-derivatives via the coefficients $\coeffA{m}{r}$,
potentially leading to a generalized notion of differentiation through limiting procedures.

