We support our theoretical findings with Wolfram~Mathematica programs that verify the main results of this manuscript.
All source code and computational notebooks are available in the
\href{https://github.com/kolosovpetro/surprising-polynomial-identities-classical-interpolation}
{\texttt{GitHub repository}}.
The repository includes the following files:
\begin{itemize}
    \item \texttt{surprising-polynomial-identities-classical-interpolation.m} --- the package file where
    all Mathematica functions are defined.
    Load it into your session using \texttt{filename.m} or by evaluating the file with \texttt{Shift+Enter}.
    \item \texttt{surprising-polynomial-identities-classical-interpolation.nb} --- a working notebook that demonstrates
    the usage of these functions to validate the manuscript’s results.
\end{itemize}

Below we list the Mathematica functions with their corresponding mathematical statements:
\begin{center}
    \renewcommand{\arraystretch}{1.4}
    \begin{tabular}{ll}
        \toprule
        \textbf{Mathematica function}                 & \textbf{Validates}                                            \\
        \midrule
        \texttt{OddPowerIdentity[n, m]}               & Theorem~\ref{theorem:odd-power-identity}                      \\
        \texttt{OddPowerIdentitySimplified[n, m]}     & Theorem~\ref{theorem:odd-power-identity} (in simplified form) \\
        \texttt{BivariateSumT[m, n, k]}               & Definition~\ref{def:bivariate-sum-Tm}                         \\
        \texttt{RecurrenceForT[m, n, k]}              & Proposition~\ref{prop:Tm-recurrence}                          \\
        \texttt{TableFormBivariateSumT[m, rows]}      & Tabular view of $T_m(n,k)$                                    \\
        \texttt{TableFormRecurrenceForT[m, rows]}     & Tabular view of the recurrence                                \\
        \texttt{OddPowerDecomposition[n, m]}          & Proposition~\ref{prop:odd-power-decomposition}                \\
        \texttt{OddPowerDecompositionMMinus1[n, m]}   & Corollary~\ref{cor:odd-power-decomposition-m-1}               \\
        \texttt{OddPowerBinomialForm[m, n, a]}        & Proposition~\ref{prop:odd-power-binomial}                     \\
        \texttt{OddPowerBinomialFormShifted[m, n, a]} & Corollary~\ref{prop:odd-power-binomial-shifted}               \\
        \bottomrule
    \end{tabular}
\end{center}
To test and experiment with these identities computationally, load the package and call any of the
functions listed above with appropriate parameters.

\subsection*{Examples of Mathematica Functions}

This subsection provides sample evaluations of the Mathematica functions
developed to validate the main results presented in the manuscript.
All computations were performed using the package \\
\texttt{surprising-polynomial-identities-classical-interpolation.m}.

\paragraph{Coefficients \texorpdfstring{$\coeffA{m}{r}$}{A(m,r)} table}
\begin{verbatim}
PrintTriangleA[5]
\end{verbatim}
\begin{align*}
    \begin{array}{l}
        \{1\} \\
        \{1,\ 6\} \\
        \{1,\ 0,\ 30\} \\
        \{1,\ -14,\ 0,\ 140\} \\
        \{1,\ -120,\ 0,\ 0,\ 630\} \\
        \{1,\ -1386,\ 660,\ 0,\ 0,\ 2772\}
    \end{array}
\end{align*}

\paragraph{Odd Power Identity}
\begin{verbatim}
OddPowerIdentity[n, 2]
\end{verbatim}
\begin{align*}
    n + n(1 + n)(-1 + n - n^2 + n^3)
\end{align*}
\begin{verbatim}
OddPowerIdentitySimplified[n, 4]
\end{verbatim}
\begin{align*}
    n^9
\end{align*}

\paragraph{Recurrence and Definition of \texorpdfstring{$T_m(n,k)$}{Tm(n,k)}}
\begin{verbatim}
RecurrenceForT[10, 3, 2]
BivariateSumT[10, 3, 2]
\end{verbatim}
\begin{align*}
    T_{10}(3,2) = 5230176601 \quad \text{(both via recurrence and explicit form)}
\end{align*}
\begin{verbatim}
Expand[BivariateSumT[2, n, k]]
\end{verbatim}
\begin{align*}
    1 + 30k^4 - 60k^3 n + 30k^2 n^2
\end{align*}
\begin{verbatim}
BivariateSumT[2, n, k]
\end{verbatim}
\begin{align*}
    1 + 30k^2(n-k)^2
\end{align*}

\paragraph{Tables of \texorpdfstring{$T_m$}{Tm}}
\begin{verbatim}
TableFormRecurrenceForT[3, 5]
TableFormBivariateSumT[3, 5]
\end{verbatim}
\begin{align*}
    \begin{array}{l}
        \{1\} \\
        \{1,\ 1\} \\
        \{1,\ 127,\ 1\} \\
        \{1,\ 1093,\ 1093,\ 1\} \\
        \{1,\ 3739,\ 8905,\ 3739,\ 1\} \\
        \{1,\ 8905,\ 30157,\ 30157,\ 8905,\ 1\}
    \end{array}
\end{align*}

\paragraph{Odd Power Decomposition}
\begin{verbatim}
OddPowerDecomposition[n, 5]
\end{verbatim}
\begin{align*}
    n^{11}
\end{align*}
\begin{verbatim}
OddPowerDecompositionMMinus1[n, 6]
\end{verbatim}
\begin{align*}
    n^{11}
\end{align*}

\paragraph{Binomial Form and Shifted Version}
\begin{verbatim}
OddPowerBinomialForm[2, 15, 5]
5^5
\end{verbatim}
\begin{align*}
    3125
\end{align*}
\begin{verbatim}
OddPowerBinomialForm[2, n, 3]
Simplify[OddPowerBinomialForm[2, n, 3]]
\end{verbatim}
\begin{align*}
    -7776 + 6480n - 2160n^2 + 360n^3 - 30n^4 + n^5 = (-6 + n)^5
\end{align*}
\begin{verbatim}
OddPowerBinomialForm[3, 16, 4]
8^7
\end{verbatim}
\begin{align*}
    2097152
\end{align*}
\begin{verbatim}
Simplify[OddPowerBinomialFormShifted[2, n, 2]]
\end{verbatim}
\begin{align*}
(-4 + n)
    ^5
\end{align*}

These examples confirm the correctness of main results of this manuscript.

