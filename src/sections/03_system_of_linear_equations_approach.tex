In 2018, Albert Tkaczyk published two papers~\cite{tkaczyk2018problem, tkaczyk2018continuation}
presenting analogous identities for polynomials $n^5, \; n^7$ and $n^9$
derived in a manner similar to $n^3 = \sum_{k=1}^{n} 6k(n-k) + 1$.
Tkaczyk assumed that the identity for $n^5$ takes the following explicit form
\begin{align*}
    n^5 = \sum_{k=1}^{n} \left[ A k^2(n-k)^2 + Bk(n-k) + C \right]
\end{align*}
where $A,B,C$ are yet-unknown coefficients.
We denote $A,B,C$ as $\coeffA{2}{0}, \coeffA{2}{1}, \coeffA{2}{2}$
to reach the compact form of double sum
\begin{align*}
    n^5 = \sum_{k=1}^{n} \sum_{r=0}^{2} \coeffA{2}{r} k^r (n-k)^r
\end{align*}
By observing the previous equation, the potential form of generalized odd-power identity becomes more obvious.
To evaluate the coefficients $\coeffA{2}{0}, \coeffA{2}{1}, \coeffA{2}{2}$
we construct and solve a certain system of linear equations, which is
built as follows
\begin{align*}
    n^5 = \coeffA{2}{0} \sum_{k=1}^{n} k^0 (n-k)^0 + \coeffA{2}{1} \sum_{k=1}^{n} k^1 (n-k)^1 + \coeffA{2}{2} \sum_{k=1}^{n} k^2 (n-k)^2
\end{align*}
By expanding the sums $\sum_{k=1}^{n} k^r (n-k)^r$ using Faulhaber's formula~\cite{beardon1996sums}, we get
an equation
\begin{equation*}
    \coeffA{m}{0} n
    + \coeffA{m}{1} \left[ \frac{1}{6} (n^3-n) \right]
    + \coeffA{m}{2} \left[ \frac{1}{30} (n^5-n) \right] - n^5 = 0
\end{equation*}
By multiplying by $30$ both right-hand side and left-hand side, we get
\begin{equation*}
    30 \coeffA{2}{0} n + 5 \coeffA{2}{1} (n^3-n) + \coeffA{2}{2} (n^5-n) - 30n^5 = 0
\end{equation*}
By expanding the brackets and rearranging the terms
\begin{equation*}
    30 \coeffA{2}{0} - 5 \coeffA{2}{1} n + 5 \coeffA{2}{1} n^3 - \coeffA{2}{2} n + \coeffA{2}{2} n^5 - 30n^5 = 0
\end{equation*}
By combining the common terms, we obtain
\begin{equation*}
    n (30 \coeffA{2}{0} - 5 \coeffA{2}{1} - \coeffA{2}{2}) + 5 \coeffA{2}{1} n^3 + n^5 (\coeffA{2}{2} - 30) = 0
\end{equation*}
Therefore,
\begin{equation*}
    \begin{cases}
        30 \coeffA{2}{0} - 5 \coeffA{2}{1} - \coeffA{2}{2} &= 0 \\
        \coeffA{2}{1} &= 0 \\
        \coeffA{2}{2} - 30 &= 0
    \end{cases}
\end{equation*}
By solving the system above, we evaluate the coefficients $\coeffA{2}{0}, \coeffA{2}{1}, \coeffA{2}{2}$
\begin{equation*}
    \begin{cases}
        \coeffA{2}{2} &= 30 \\
        \coeffA{2}{1} &= 0 \\
        \coeffA{2}{0} &= 1
    \end{cases}
\end{equation*}
Thus, the identity for $n^5$
\begin{equation*}
    n^5 = \sum_{k=1}^{n} 30k^2(n-k)^2 + 1
\end{equation*}
Again, the terms $30k^2(n-k)^2 + 1$ are symmetric over $k$.
Let be $T_2 (n,k) = 30k^2(n-k)^2 + 1$ then
\begin{align*}
    T_2 (n,k) = T_2 (n,n-k)
\end{align*}
By arranging the values of $T_{2} (n,k)$ as a triangular array, we see that the identity for $n^5$ is indeed true
\begin{table}[H]
    \setlength\extrarowheight{-6pt}
    \begin{tabular}{c|cccccccc}
        $n/k$ & 0 & 1    & 2    & 3    & 4    & 5    & 6    & 7 \\
        \hline
        0     & 1 &      &      &      &      &      &      &   \\
        1     & 1 & 1    &      &      &      &      &      &   \\
        2     & 1 & 31   & 1    &      &      &      &      &   \\
        3     & 1 & 121  & 121  & 1    &      &      &      &   \\
        4     & 1 & 271  & 481  & 271  & 1    &      &      &   \\
        5     & 1 & 481  & 1081 & 1081 & 481  & 1    &      &   \\
        6     & 1 & 751  & 1921 & 2431 & 1921 & 751  & 1    &   \\
        7     & 1 & 1081 & 3001 & 4321 & 4321 & 3001 & 1081 & 1
    \end{tabular}
    \caption{Values of $30k^2(n-k)^2 + 1$.
    See the OEIS entry \href{https://oeis.org/A300656}{\texttt{A300656}}
    ~\cite{oeis_numerical_triangle_row_sums_give_fifth_powers}.}
    \label{tab:row-sums-gives-fifth-power}
\end{table}

The following recurrence holds for $T_2 (n,k)$
\begin{align*}
    T_2 (n, k) = 3T_2(n-1, k) - 3T_2(n-2, k) + T_2(n-3, k)
\end{align*}
Which is indeed true because
\begin{align*}
    T_2 (6,2) = 3 \cdot 1080 - 3 \cdot 481 + 271 = 1921
\end{align*}

Consider an example for $n^7$.
\begin{align*}
    n^7 =
    \coeffA{m}{0} n
    + \coeffA{m}{1} \left[ \frac{1}{6} (n^3-n) \right]
    &+ \coeffA{m}{2} \left[ \frac{1}{30} (n^5-n) \right] \\
    &+ \coeffA{m}{3} \left[ \frac{1}{420} (3n^7+7n^3-10n) \right]
\end{align*}
By multiplying by $30$ both right-hand side and left-hand side, we get
\begin{align*}
    420 \coeffA{3}{0} n + 70 \coeffA{2}{1} (n^3-n)
    &+ 14 \coeffA{2}{2} (n^5-n) \\
    &+ \coeffA{3}{3} (3n^7+7n^3-10n) - 420n^7 = 0
\end{align*}
By expanding the brackets and rearranging terms, we obtain
\begin{align*}
    420 \coeffA{3}{0} n
    &- 70 \coeffA{3}{1} + 70 \coeffA{3}{1} n^3 - 14 \coeffA{3}{2} n + 14 \coeffA{3}{2} n^5 \\
    &- 10 \coeffA{3}{3} n + 7 \coeffA{3}{3} n^3 + 3 \coeffA{3}{3} n^7 - 420n^7 = 0
\end{align*}
Combining the common terms yields
\begin{align*}
    n (420 \coeffA{3}{0} - 70 \coeffA{3}{1} - 14 \coeffA{3}{2} - 10 \coeffA{3}{3})
    &+ n^3 (70 \coeffA{3}{1} + 7 \coeffA{3}{3}) \\
    &+ n^5 14 \coeffA{3}{2}
    + n^7 (3 \coeffA{3}{3} - 420)
    = 0
\end{align*}
Therefore, we receive the following system of linear equations
\begin{align*}
    \begin{cases}
        420 \coeffA{3}{0} - 70 \coeffA{3}{1} - 14 \coeffA{3}{2} - 10 \coeffA{3}{3} &= 0 \\
        70 \coeffA{3}{1} + 7 \coeffA{3}{3} &= 0 \\
        \coeffA{3}{2} - 30 &= 0 \\
        3 \coeffA{3}{3} - 420 &= 0
    \end{cases}
\end{align*}
By solving it for $\coeffA{3}{0}, \coeffA{3}{1}, \coeffA{3}{2}, \coeffA{3}{3}$, we obtain
\begin{equation*}
    \begin{cases}
        \coeffA{3}{3} &= 140 \\
        \coeffA{3}{2} &= 0 \\
        \coeffA{3}{1} = -\frac{7}{70} \coeffA{3}{3} &= -14 \\
        \coeffA{3}{0} = \frac{(70 \coeffA{3}{1} + 10 \coeffA{3}{3})}{420} &= 1
    \end{cases}
\end{equation*}
Therefore, the identity for $n^7$ holds
\begin{equation*}
    n^7 = \sum_{k=1}^{n} 140 k^3 (n-k)^3 - 14k(n-k) + 1
\end{equation*}
The values of $140 k^3 (n-k)^3 - 14k(n-k) + 1$ are symmetric over $k$.
Let be $T_3 (n,k) = 140 k^3 (n-k)^3 - 14k(n-k) + 1$ then
\begin{align*}
    T_3 (n,k) = T_3 (n, n-k)
\end{align*}
By arranging the values of $T_{3} (n,k)$ as a triangular array, we see that the identity for $n^7$ is indeed true
\begin{table}[H]
    \setlength\extrarowheight{-6pt}
    \begin{tabular}{c|cccccccc}
        $n/k$ & 0 & 1     & 2      & 3      & 4      & 5      & 6     & 7 \\
        \hline
        0     & 1 &       &        &        &        &        &       &   \\
        1     & 1 & 1     &        &        &        &        &       &   \\
        2     & 1 & 127   & 1      &        &        &        &       &   \\
        3     & 1 & 1093  & 1093   & 1      &        &        &       &   \\
        4     & 1 & 3739  & 8905   & 3739   & 1      &        &       &   \\
        5     & 1 & 8905  & 30157  & 30157  & 8905   & 1      &       &   \\
        6     & 1 & 17431 & 71569  & 101935 & 71569  & 17431  & 1     &   \\
        7     & 1 & 30157 & 139861 & 241753 & 241753 & 139861 & 30157 & 1
    \end{tabular}
    \caption{Values of $140 k^3 (n-k)^3 - 14k(n-k) + 1$.
    See the OEIS entry \href{https://oeis.org/A300785}{\texttt{A300785}}
    ~\cite{oeis_numerical_triangle_row_sums_give_seventh_powers}.}
    \label{tab:row-sums-gives-seventh-power}
\end{table}

The following recurrence holds for $T_{3} (n,k)$
\begin{align*}
    T_{3} (n, k) = 4T_{3} (n-1, k) - 6T_{3} (n-2, k) + 4T_{3} (n-3, k) - T_{3} (n-4, k)
\end{align*}
Which is true indeed because
\begin{align*}
    T_{3} (7, 1) = 4 \cdot 17431 -6 \cdot 8905 + 4 \cdot 3739 - 1 \cdot 1093 = 30157
\end{align*}
